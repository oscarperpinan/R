% Created 2014-08-27 mié 09:21
\documentclass[xcolor={usenames,svgnames,dvipsnames}]{beamer}
\usepackage[utf8]{inputenc}
\usepackage[T1]{fontenc}
\usepackage{fixltx2e}
\usepackage{graphicx}
\usepackage{longtable}
\usepackage{float}
\usepackage{wrapfig}
\usepackage{rotating}
\usepackage[normalem]{ulem}
\usepackage{amsmath}
\usepackage{textcomp}
\usepackage{marvosym}
\usepackage{wasysym}
\usepackage{amssymb}
\usepackage{hyperref}
\tolerance=1000
\usepackage{color}
\usepackage{listings}
\AtBeginSection[]{\begin{frame}[plain]\tableofcontents[currentsection,hideallsubsections]\end{frame}}
\lstset{keywordstyle=\color{blue}, commentstyle=\color{gray!90}, basicstyle=\ttfamily\small, columns=fullflexible, breaklines=true,linewidth=\textwidth, backgroundcolor=\color{gray!23}, basewidth={0.5em,0.4em}, literate={á}{{\'a}}1 {ñ}{{\~n}}1 {é}{{\'e}}1 {ó}{{\'o}}1 {º}{{\textordmasculine}}1}
\usepackage{mathpazo}
\hypersetup{colorlinks=true, linkcolor=Blue, urlcolor=Blue}
\usepackage{fancyvrb}
\DefineVerbatimEnvironment{verbatim}{Verbatim}{fontsize=\tiny, formatcom = {\color{black!70}}}
\usetheme{Goettingen}
\usecolortheme{rose}
\usefonttheme{serif}
\author{Oscar Perpiñán Lamigueiro \\ \url{http://oscarperpinan.github.io}}
\date{}
\title{Series temporales con zoo}
\hypersetup{
  pdfkeywords={},
  pdfsubject={},
  pdfcreator={Emacs 24.3.1 (Org mode 8.2.1)}}
\begin{document}

\maketitle


\section{Lectura de ficheros (sencillo)}
\label{sec-1}
\begin{frame}[label=sec-1-1]{Descargamos datos de SIAR}
\begin{itemize}
\item \url{http://eportal.magrama.gob.es/websiar/Inicio.aspx}
\item \alert{Estación}: Aranjuez, Madrid
\item \alert{Período}: 01/01/2004 a 31/12/2011
\item \alert{Variables}: Temperatura, Humedad, Viento, Lluvia, Radiación, ET
\end{itemize}
\end{frame}
\begin{frame}[fragile,label=sec-1-2]{Lectura de datos con \texttt{read.table}}
 \begin{itemize}
\item Primero lo intentamos con la versión final
\end{itemize}
\lstset{language=R,numbers=none}
\begin{lstlisting}
dats <- read.table('data/aranjuez.csv')
head(dats)

dats <- read.table('data/aranjuez.csv', sep=',')
head(dats)

dats <- read.table('data/aranjuez.csv', sep=',', header=TRUE)
head(dats)

aranjuez <- read.csv('data/aranjuez.csv')
head(aranjuez)

class(aranjuez)
names(aranjuez)
\end{lstlisting}
\end{frame}
\begin{frame}[fragile,label=sec-1-3]{Visualización de datos}
 \lstset{language=R,numbers=none}
\begin{lstlisting}
library(lattice)

xyplot(Radiation ~ TempAvg, data=aranjuez)

xyplot(Radiation ~ TempAvg, data=aranjuez,
       type=c('p', 'r'))

xyplot(Radiation ~ TempAvg + TempMax + TempMin,
       data = aranjuez, xlab='Temperature',
       type=c('p', 'r'), auto.key=TRUE,
       pch=16, alpha=0.5)
\end{lstlisting}
\end{frame}
\begin{frame}[fragile,label=sec-1-4]{Visualización de datos (advanced!)}
 \lstset{language=R,numbers=none}
\begin{lstlisting}
library(RColorBrewer)

humidClass <- cut(aranjuez$HumidAvg, 4)
myPal <- brewer.pal(n=4, 'GnBu')

xyplot(Radiation ~ TempAvg + TempMax + TempMin,
       groups=humidClass, outer=TRUE,
       data = aranjuez, xlab='Temperature',
       layout=c(3, 1),
       scales=list(relation='free'),
       auto.key=list(space='right'),
       par.settings=custom.theme(pch=16,
	 alpha=0.8, col=myPal))
\end{lstlisting}
\end{frame}

\begin{frame}[fragile,label=sec-1-5]{Transformamos a serie temporal}
 \lstset{language=R,numbers=none}
\begin{lstlisting}
library(zoo)

fecha <- as.POSIXct(aranjuez[,1],
		    format='%Y-%m-%d')
head(fecha)

aranjuez <- zoo(aranjuez[, -1], fecha)
class(aranjuez)
head(aranjuez)
\end{lstlisting}
\end{frame}
\begin{frame}[fragile,label=sec-1-6]{Leemos directamente como serie temporal}
 \lstset{language=R,numbers=none}
\begin{lstlisting}
aranjuez <- read.zoo('data/aranjuez.csv',
		     sep=',', header=TRUE)
\end{lstlisting}

\lstset{language=R,numbers=none}
\begin{lstlisting}
header(aranjuez)
names(aranjuez)
summary(index(aranjuez))
\end{lstlisting}
\end{frame}
\section{Lectura de datos (real)}
\label{sec-2}
\begin{frame}[fragile,label=sec-2-1]{Ahora con la versión original}
 \begin{itemize}
\item Primero descomprimimos el archivo
\end{itemize}
\lstset{language=R,numbers=none}
\begin{lstlisting}
unzip('data/InformeDatos.zip', exdir='data')
\end{lstlisting}
\begin{itemize}
\item Y ahora abrimos teniendo en cuenta codificación, separadores, etc.
\end{itemize}
\lstset{language=R,numbers=none}
\begin{lstlisting}
aranjuez <- read.table("data/M03_Aranjuez_01_01_2004_31_12_2011.csv",
		     fileEncoding = 'UTF-16LE',
		     header = TRUE, fill = TRUE,
		     sep = ';', dec = ",")
\end{lstlisting}
\begin{itemize}
\item Vemos el contenido
\end{itemize}
\lstset{language=R,numbers=none}
\begin{lstlisting}
head(aranjuez)
summary(aranjuez)
names(aranjuez)
\end{lstlisting}
\end{frame}
\begin{frame}[fragile,label=sec-2-2]{Convertimos a serie temporal}
 \begin{itemize}
\item Sólo nos interesan algunas variables (indexamos por columnas)
\end{itemize}
\lstset{language=R,numbers=none}
\begin{lstlisting}
tt <- as.Date(aranjuez$Fecha, format='%d/%m/%Y')
aranjuez <- zoo(aranjuez[, c(6, 7, 9, 11, 12, 16,
			     17, 19, 20, 22)],
		order.by=tt)
\end{lstlisting}
\end{frame}
\begin{frame}[fragile,label=sec-2-3]{Ajustamos los nombres (opcional)}
 \lstset{language=R,numbers=none}
\begin{lstlisting}
names(aranjuez) <- c('TempAvg', 'TempMax',
		     'TempMin', 'HumidAvg',
		     'HumidMax','WindAvg',
		     'WindMax', 'Radiation',
		     'Rain', 'ET')
\end{lstlisting}
\end{frame}
\begin{frame}[fragile,label=sec-2-4]{Nuevamente mostramos datos}
 \begin{itemize}
\item Método simple
\end{itemize}
\lstset{language=R,numbers=none}
\begin{lstlisting}
xyplot(aranjuez)
\end{lstlisting}
\begin{itemize}
\item Seleccionamos variables y superponemos
\end{itemize}
\lstset{language=R,numbers=none}
\begin{lstlisting}
xyplot(aranjuez[,c("TempAvg", "TempMax", "TempMin")],
       superpose=TRUE)
\end{lstlisting}
\begin{itemize}
\item Para cruzar variables hay que convertir a \texttt{data.frame}
\end{itemize}
\lstset{language=R,numbers=none}
\begin{lstlisting}
xyplot(TempAvg ~ Radiation,
       data=as.data.frame(aranjuez))
\end{lstlisting}
\end{frame}
\begin{frame}[fragile,label=sec-2-5]{Limpieza de datos}
 \begin{itemize}
\item Conversión de Unidades (MJ -> Wh)
\end{itemize}
\lstset{language=R,numbers=none}
\begin{lstlisting}
aranjuez$G0 <- aranjuez$Radiation/3.6*1000
xyplot(aranjuez$G0)
\end{lstlisting}
\begin{itemize}
\item Filtrado de datos
\end{itemize}
\lstset{language=R,numbers=none}
\begin{lstlisting}
aranjuezClean <- within(as.data.frame(aranjuez),{
  TempMin[TempMin>40] <- NA
  HumidMax[HumidMax>100] <- NA
  WindAvg[WindAvg>10] <- NA
  WindMax[WindMax>10] <- NA
})

aranjuez <- zoo(aranjuezClean, index(aranjuez))
\end{lstlisting}
\end{frame}
\section{Datos agregados}
\label{sec-3}
\begin{frame}[fragile,label=sec-3-1]{Media anual}
 \begin{itemize}
\item Primero definimos una función para extraer el año
\end{itemize}
\lstset{language=R,numbers=none}
\begin{lstlisting}
Year <- function(x)as.numeric(format(x, "%Y"))

Year(index(aranjuez))
\end{lstlisting}
\begin{itemize}
\item Y la empleamos para agrupar con \texttt{aggregate}
\end{itemize}
\lstset{language=R,numbers=none}
\begin{lstlisting}
aranjuezY <- aggregate(aranjuez$G0, by=Year,
		       FUN=mean, na.rm=TRUE)
aranjuezY
class(aranjuezY)
\end{lstlisting}

\lstset{language=R,numbers=none}
\begin{lstlisting}
G0y <- aggregate(aranjuez$G0, by=Year,
		 FUN=mean, na.rm=TRUE)
G0y
\end{lstlisting}
\end{frame}
\begin{frame}[fragile,label=sec-3-2]{Medias anuales usando \texttt{cut}}
 \lstset{language=R,numbers=none}
\begin{lstlisting}
aggregate(aranjuez$G0, by=function(tt)cut(tt, 'year'),
	  FUN=mean, na.rm=TRUE)
\end{lstlisting}
\end{frame}
\begin{frame}[fragile,label=sec-3-3]{Medias mensuales}
 \begin{itemize}
\item Meses como números
\end{itemize}
\lstset{language=R,numbers=none}
\begin{lstlisting}
Month <- function(x)as.numeric(format(x, "%m"))

Month(index(aranjuez))
\end{lstlisting}

\lstset{language=R,numbers=none}
\begin{lstlisting}
G0m <- aggregate(aranjuez$G0, by=Month,
		 FUN=mean, na.rm=TRUE)
G0m
\end{lstlisting}

\begin{itemize}
\item Meses como etiquetas
\end{itemize}
\lstset{language=R,numbers=none}
\begin{lstlisting}
months(index(aranjuez))
\end{lstlisting}

\lstset{language=R,numbers=none}
\begin{lstlisting}
G0m <- aggregate(aranjuez$G0, by=months,
		 FUN=mean, na.rm=TRUE)
G0m
\end{lstlisting}
\end{frame}
\begin{frame}[fragile,label=sec-3-4]{Medias mensuales para cada año}
 \begin{itemize}
\item La función para agrupar es \texttt{as.yearmon}
\end{itemize}
\lstset{language=R,numbers=none}
\begin{lstlisting}
as.yearmon(index(aranjuez))
\end{lstlisting}

\lstset{language=R,numbers=none}
\begin{lstlisting}
G0ym <- aggregate(aranjuez$G0, by=as.yearmon,
		  FUN=mean, na.rm=TRUE)
G0ym
\end{lstlisting}
\end{frame}
\section{Datos desde una URL}
\label{sec-4}
\begin{frame}[fragile,label=sec-4-1]{Ejemplo: Lanai-Hawaii}
 \lstset{language=R,numbers=none}
\begin{lstlisting}
URL <- "http://www.nrel.gov/midc/apps/plot.pl?site=LANAI&start=20090722&edy=19&emo=11&eyr=2010&zenloc=19&year=2010&month=11&day=1&endyear=2010&endmonth=11&endday=19&time=1&inst=3&inst=4&inst=5&inst=10&type=data&first=3&math=0&second=-1&value=0.0&global=-1&direct=-1&diffuse=-1&user=0&axis=1"
## URL <- "data/NREL-Hawaii.csv"
\end{lstlisting}

\begin{verbatim}
DATE,HST,Global Horizontal [W/m^2],Direct Normal [W/m^2],Diffuse Horizontal [W/m^2],Air Temperature [deg C]
11/1/2010,06:32,4.87621,0,4.87621,14.67
11/1/2010,06:33,5.14142,0,5.14142,14.54
11/1/2010,06:34,1.42216,0,1.42216,14.43
11/1/2010,06:35,1.95135,0,1.95135,14.4
11/1/2010,06:36,2.44687,0,2.44687,14.55
11/1/2010,06:37,3.16990,0,3.16990,14.95
11/1/2010,06:38,3.99677,0,3.99677,15.45
11/1/2010,06:39,4.88811,0,4.88811,15.71
11/1/2010,06:40,5.85428,0,5.85428,15.8
11/1/2010,06:41,8.27598,0,8.27598,15.87
\end{verbatim}
\end{frame}
\begin{frame}[fragile,label=sec-4-2]{Leemos como serie temporal}
 \begin{itemize}
\item Leemos con \texttt{read.zoo}
\end{itemize}
\lstset{language=R,numbers=none}
\begin{lstlisting}
lat <- 20.77
lon <- -156.9339
hawaii <- read.zoo(URL,
		col.names = c("date", "hour",
		  "G0", "B", "D0", "Ta"),
		## Dia en columna 1, Hora en columna 2
		index = list(1, 2),
		## Obtiene escala temporal de estas dos columnas
		FUN = function(d, h) as.POSIXct(
		  paste(d, h),
		  format = "%m/%d/%Y %H:%M",
		  tz = "HST"), 
		header=TRUE, sep=",")
\end{lstlisting}
\begin{itemize}
\item Añadimos Directa en el plano Horizontal
\end{itemize}
\lstset{language=R,numbers=none}
\begin{lstlisting}
hawaii$B0 <- with(hawaii, G0-D0)
\end{lstlisting}
\end{frame}
\begin{frame}[fragile,label=sec-4-3]{Mostramos datos como serie temporal}
 \lstset{language=R,numbers=none}
\begin{lstlisting}
xyplot(hawaii)
xyplot(hawaii[,c('G0', 'D0', 'B0')],
       superpose=TRUE)
\end{lstlisting}
\end{frame}
\begin{frame}[fragile,label=sec-4-4]{Mostramos relaciones entre variables}
 \lstset{language=R,numbers=none}
\begin{lstlisting}
xyplot(Ta ~ G0 + D0 + B0,
       data=as.data.frame(hawaii),
       type=c('p', 'smooth'),
       par.settings=custom.theme(
	 alpha=.5, pch=16,
	 lwd=3, col.line='black'),
       outer=TRUE, layout=c(3, 1),
       scales=list(x=list(relation='free')))
\end{lstlisting}
\end{frame}
\begin{frame}[fragile,label=sec-4-5]{Irradiación horaria}
 \begin{itemize}
\item Primer intento
\end{itemize}
\lstset{language=R,numbers=none}
\begin{lstlisting}
hour <- function(x)as.numeric(format(x, '%H'))
\end{lstlisting}

\lstset{language=R,numbers=none}
\begin{lstlisting}
G0h <- aggregate(hawaii$G0, by=hour,
		 FUN=sum, na.rm=1)/1000
G0h
\end{lstlisting}
\end{frame}
\begin{frame}[fragile,label=sec-4-6]{Irradiación horaria}
 \begin{itemize}
\item Mejor así
\end{itemize}
\lstset{language=R,numbers=none}
\begin{lstlisting}
hour <- function(x)as.POSIXct(format(x,
				     '%Y-%m-%d %H:00:00'))
\end{lstlisting}

\lstset{language=R,numbers=none}
\begin{lstlisting}
G0h <- aggregate(hawaii$G0, by=hour,
		 FUN=sum, na.rm=1)/60
G0h
\end{lstlisting}
\end{frame}
\begin{frame}[fragile,label=sec-4-7]{Irradiación diaria}
 \begin{itemize}
\item A partir de la horaria
\end{itemize}
\lstset{language=R,numbers=none}
\begin{lstlisting}
G0d <- aggregate(G0h,
		 by=function(x)format(x, '%Y-%m-%d'),
		 sum)/1000
\end{lstlisting}
\begin{itemize}
\item A partir de la minutaria
\end{itemize}
\lstset{language=R,numbers=none}
\begin{lstlisting}
day <- function(x)format(x, '%Y-%m-%d')
G0d <- aggregate(hawaii$G0, by=day,
		 sum)/60/1000
G0d

truncDay <- function(x)as.POSIXct(trunc(x, units='day'))
G0d <- aggregate(hawaii$G0, by=truncDay,
		 sum)/60/1000
G0d
\end{lstlisting}
\end{frame}

\begin{frame}[fragile,label=sec-4-8]{Más complicado: agrupar por 30 minutos}
 \lstset{language=R,numbers=none}
\begin{lstlisting}
halfHour <- function(tt, delta=30){
  tt <- as.POSIXlt(tt)
  gg <- tt$min %/% delta
  tt <- modifyList(tt, list(min=gg*delta))
  as.POSIXct(tt)
}
\end{lstlisting}

\lstset{language=R,numbers=none}
\begin{lstlisting}
hawaii30 <- aggregate(hawaii, by=halfHour,
		      FUN=sum)/60

head(hawaii30)
\end{lstlisting}
\end{frame}
% Emacs 24.3.1 (Org mode 8.2.1)
\end{document}