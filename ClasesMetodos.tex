% Created 2014-08-27 mié 08:36
\documentclass[xcolor={usenames,svgnames,dvipsnames}]{beamer}
\usepackage[utf8]{inputenc}
\usepackage[T1]{fontenc}
\usepackage{fixltx2e}
\usepackage{graphicx}
\usepackage{longtable}
\usepackage{float}
\usepackage{wrapfig}
\usepackage{rotating}
\usepackage[normalem]{ulem}
\usepackage{amsmath}
\usepackage{textcomp}
\usepackage{marvosym}
\usepackage{wasysym}
\usepackage{amssymb}
\usepackage{hyperref}
\tolerance=1000
\usepackage{color}
\usepackage{listings}
\AtBeginSection[]{\begin{frame}[plain]\tableofcontents[currentsection,hideallsubsections]\end{frame}}
\lstset{keywordstyle=\color{blue}, commentstyle=\color{gray!90}, basicstyle=\ttfamily\small, columns=fullflexible, breaklines=true,linewidth=\textwidth, backgroundcolor=\color{gray!23}, basewidth={0.5em,0.4em}, literate={á}{{\'a}}1 {ñ}{{\~n}}1 {é}{{\'e}}1 {ó}{{\'o}}1 {º}{{\textordmasculine}}1}
\usepackage{mathpazo}
\hypersetup{colorlinks=true, linkcolor=Blue, urlcolor=Blue}
\usepackage{fancyvrb}
\DefineVerbatimEnvironment{verbatim}{Verbatim}{fontsize=\tiny, formatcom = {\color{black!70}}}
\usetheme{Goettingen}
\usecolortheme{rose}
\usefonttheme{serif}
\author{Oscar Perpiñán Lamigueiro \\ \url{http://oscarperpinan.github.io}}
\date{}
\title{Clases y Métodos}
\hypersetup{
  pdfkeywords={},
  pdfsubject={},
  pdfcreator={Emacs 24.3.1 (Org mode 8.2.1)}}
\begin{document}

\maketitle

\section{OOP en R}
\label{sec-1}
\subsection{Programación Orientada a Objetos (OOP)}
\label{sec-1-1}

\begin{frame}[fragile,label=sec-1-1-1]{Programación Orientada a Objetos (OOP)}
 \begin{itemize}
\item Características básicas del paradigma OOP:
\begin{itemize}
\item Los objectos encapsulan información y control de su comportamiento (\emph{objects}).
\item Las clases describen propiedades de un grupo de objetos (\emph{class}).
\item Se pueden definir clases a partir de otras (\emph{inheritance}).
\item Una función genérica se comporta de forma diferente atendiendo a la
clase de uno (o varios) de sus argumentos (\emph{polymorphism}).
\end{itemize}
\item En \texttt{R} coexisten dos implementaciones de la OOP:
\begin{itemize}
\item \texttt{S3}: elaboración informal con enfasis en las funciones genéricas y el polimorfismo.
\item \texttt{S4}: elaboración formal de clases y métodos.
\end{itemize}
\end{itemize}
\end{frame}

\begin{frame}[label=sec-1-1-2]{OOP en R}
\begin{block}{Referencias}
\begin{center}
\begin{itemize}
\item \href{http://books.google.es/books/about/Software_for_Data_Analysis.html}{Software for Data Analysis}
\item \href{http://developer.r-project.org/howMethodsWork.pdf}{How Methods Work}
\item \href{http://www.stat.auckland.ac.nz/S-Workshop/Gentleman/S4Objects.pdf}{S4 classes in 15 pages}
\item \href{http://bioconductor.org/help/publications/books/r-programming-for-bioinformatics/}{R Programming for Bioinformatics }
\item \href{http://bioconductor.org/help/course-materials/2010/AdvancedR/S4InBioconductor.pdf}{S4 System Development in Bioconductor}
\end{itemize}
\end{center}
\end{block}
\end{frame}
\section{Clases y métodos S3}
\label{sec-2}

\subsection{Clases}
\label{sec-2-1}
\begin{frame}[fragile,label=sec-2-1-1]{Clases}
 \begin{itemize}
\item Los objetos básicos en \texttt{R} tienen una clase implícita definida en \texttt{S3}. Es accesible con \texttt{class}.
\end{itemize}
\lstset{language=R,numbers=none}
\begin{lstlisting}
x <- rnorm(10)
class(x)
\end{lstlisting}

\begin{verbatim}
[1] "numeric"
\end{verbatim}

\begin{itemize}
\item Pero no tienen atributo ni se consideran formalmente objetos:
\end{itemize}
\lstset{language=R,numbers=none}
\begin{lstlisting}
attr(x, 'class')
\end{lstlisting}

\begin{verbatim}
NULL
\end{verbatim}

\lstset{language=R,numbers=none}
\begin{lstlisting}
is.object(x)
\end{lstlisting}

\begin{verbatim}
[1] FALSE
\end{verbatim}
\end{frame}

\begin{frame}[fragile,label=sec-2-1-2]{Clases}
 \begin{itemize}
\item Se puede redefinir la clase de un objecto \texttt{S3} con \texttt{class}
\end{itemize}
\lstset{language=R,numbers=none}
\begin{lstlisting}
class(x) <- 'myNumeric'
class(x)
\end{lstlisting}

\begin{verbatim}
[1] "myNumeric"
\end{verbatim}

\begin{itemize}
\item Ahora sí es un objeto y su atributo está definido:
\end{itemize}
\lstset{language=R,numbers=none}
\begin{lstlisting}
attr(x, 'class')
\end{lstlisting}

\begin{verbatim}
[1] "myNumeric"
\end{verbatim}

\lstset{language=R,numbers=none}
\begin{lstlisting}
is.object(x)
\end{lstlisting}

\begin{verbatim}
[1] TRUE
\end{verbatim}

\begin{itemize}
\item Sin embargo, su modo de almacenamiento (clase intrínseca) no cambia:
\end{itemize}
\lstset{language=R,numbers=none}
\begin{lstlisting}
mode(x)
\end{lstlisting}

\begin{verbatim}
[1] "numeric"
\end{verbatim}
\end{frame}
\begin{frame}[fragile,label=sec-2-1-3]{Definición de Clases}
 \lstset{language=R,numbers=none}
\begin{lstlisting}
task1 <- list(what='Write an email',
	      when=as.Date('2013-01-01'),
	      priority='Low')
class(task1) <- 'task3'
task1
\end{lstlisting}

\begin{verbatim}
$what
[1] "Write an email"

$when
[1] "2013-01-01"

$priority
[1] "Low"

attr(,"class")
[1] "task3"
\end{verbatim}

\lstset{language=R,numbers=none}
\begin{lstlisting}
task2 <- list(what='Find and fix bugs',
	      when=as.Date('2013-03-15'),
	      priority='High')
class(task2) <- 'task3'
\end{lstlisting}
\end{frame}
\begin{frame}[fragile,label=sec-2-1-4]{Definición de Clases}
 \lstset{language=R,numbers=none}
\begin{lstlisting}
myToDo <- list(task1, task2)
class(myToDo) <- c('ToDo3')
myToDo
\end{lstlisting}

\begin{verbatim}
[[1]]
$what
[1] "Write an email"

$when
[1] "2013-01-01"

$priority
[1] "Low"

attr(,"class")
[1] "task3"

[[2]]
$what
[1] "Find and fix bugs"

$when
[1] "2013-03-15"

$priority
[1] "High"

attr(,"class")
[1] "task3"

attr(,"class")
[1] "ToDo3"
\end{verbatim}
\end{frame}
\subsection{Métodos con \texttt{S3}}
\label{sec-2-2}

\begin{frame}[fragile,label=sec-2-2-1]{Métodos con \texttt{S3}: \texttt{NextMethod}}
 \lstset{language=R,numbers=none}
\begin{lstlisting}
print.task3 <- function(x, ...){
  cat('Task:\n')
  NextMethod(x, ...)
}
\end{lstlisting}

\lstset{language=R,numbers=none}
\begin{lstlisting}
print(task1)
\end{lstlisting}

\begin{verbatim}
Task:
$what
[1] "Write an email"

$when
[1] "2013-01-01"

$priority
[1] "Low"

attr(,"class")
[1] "task3"
\end{verbatim}
\end{frame}
\begin{frame}[fragile,label=sec-2-2-2]{Métodos con \texttt{S3}: \texttt{NextMethod}}
 \lstset{language=R,numbers=none}
\begin{lstlisting}
print.ToDo3 <- function(x, ...){
  cat('This is my ToDo list:\n')
  NextMethod(x, ...)
  cat('--------------------\n')
}
\end{lstlisting}

\lstset{language=R,numbers=none}
\begin{lstlisting}
print(myToDo)
\end{lstlisting}

\begin{verbatim}
This is my ToDo list:
[[1]]
Task:
$what
[1] "Write an email"

$when
[1] "2013-01-01"

$priority
[1] "Low"

attr(,"class")
[1] "task3"

[[2]]
Task:
$what
[1] "Find and fix bugs"

$when
[1] "2013-03-15"

$priority
[1] "High"

attr(,"class")
[1] "task3"

attr(,"class")
[1] "ToDo3"
--------------------
\end{verbatim}
\end{frame}

\begin{frame}[fragile,label=sec-2-2-3]{Definición de un método \texttt{S3} para \texttt{ToDo3}}
 \lstset{language=R,numbers=none}
\begin{lstlisting}
print.ToDo3 <- function(x, ...){
  cat('This is my ToDo list:\n')
  for (i in seq_along(x)){
    cat('Task no.', i,':\n')
    cat('What: ', x[[i]]$what,
	'- When:', as.character(x[[i]]$when),
	'- Priority:', x[[i]]$priority,
	'\n')
    }
  cat('--------------------\n')
}
\end{lstlisting}

\lstset{language=R,numbers=none}
\begin{lstlisting}
print(myToDo)
\end{lstlisting}

\begin{verbatim}
This is my ToDo list:
Task no. 1 :
What:  Write an email - When: 2013-01-01 - Priority: Low 
Task no. 2 :
What:  Find and fix bugs - When: 2013-03-15 - Priority: High 
--------------------
\end{verbatim}
\end{frame}
\begin{frame}[fragile,label=sec-2-2-4]{Definición de un método \texttt{S3} para \texttt{task3}}
 \lstset{language=R,numbers=none}
\begin{lstlisting}
print.task3 <- function(x, number,...){
  if (!missing(number)) cat('Task no.', number,':\n')
  cat('What: ', x$what,
      '- When:', as.character(x$when),
      '- Priority:', x$priority,
      '\n')
}
\end{lstlisting}

\lstset{language=R,numbers=none}
\begin{lstlisting}
print(task1)
\end{lstlisting}

\begin{verbatim}
What:  Write an email - When: 2013-01-01 - Priority: Low
\end{verbatim}

\lstset{language=R,numbers=none}
\begin{lstlisting}
print(myToDo[[2]])
\end{lstlisting}

\begin{verbatim}
What:  Find and fix bugs - When: 2013-03-15 - Priority: High
\end{verbatim}
\end{frame}
\begin{frame}[fragile,label=sec-2-2-5]{Redefinición del método para \texttt{ToDo3}}
 \lstset{language=R,numbers=none}
\begin{lstlisting}
print.ToDo3 <- function(x, ...){
  cat('This is my ToDo list:\n')
  for (i in seq_along(x)) print(x[[i]], i)
    cat('--------------------\n')
}
\end{lstlisting}

\lstset{language=R,numbers=none}
\begin{lstlisting}
print(myToDo)
\end{lstlisting}

\begin{verbatim}
This is my ToDo list:
Task no. 1 :
What:  Write an email - When: 2013-01-01 - Priority: Low 
Task no. 2 :
What:  Find and fix bugs - When: 2013-03-15 - Priority: High 
--------------------
\end{verbatim}
\end{frame}
\subsection{Métodos genéricos con \texttt{S3}}
\label{sec-2-3}
\begin{frame}[fragile,label=sec-2-3-1]{Métodos genéricos con \texttt{S3}: \texttt{UseMethod}}
 \lstset{language=R,numbers=none}
\begin{lstlisting}
myFun <- function(x, ...)UseMethod('myFun')
myFun.default <- function(x, ...){
  cat('Funcion genérica\n')
  print(x)
  }
\end{lstlisting}

\lstset{language=R,numbers=none}
\begin{lstlisting}
myFun(x)
\end{lstlisting}

\begin{verbatim}
Funcion genérica
 [1] -1.2955355 -1.6192261  1.6532686  0.5235297  0.3161940 -1.6700883
 [7] -0.1509467 -3.1920897 -0.1848521  0.5497050
attr(,"class")
[1] "myNumeric"
\end{verbatim}

\lstset{language=R,numbers=none}
\begin{lstlisting}
myFun(task1)
\end{lstlisting}

\begin{verbatim}
Funcion genérica
What:  Write an email - When: 2013-01-01 - Priority: Low
\end{verbatim}
\end{frame}
\begin{frame}[fragile,label=sec-2-3-2]{\texttt{methods}}
 \begin{itemize}
\item Con \texttt{methods} podemos averiguar los métodos que hay definidos para una función particular:
\end{itemize}
\lstset{language=R,numbers=none}
\begin{lstlisting}
methods('myFun')
\end{lstlisting}

\begin{verbatim}
[1] myFun.default
\end{verbatim}

\lstset{language=R,numbers=none}
\begin{lstlisting}
head(methods('print'))
\end{lstlisting}

\begin{verbatim}
[1] "print.acf"     "print.anova"   "print.aov"     "print.aovlist"
[5] "print.ar"      "print.Arima"
\end{verbatim}
\end{frame}
\begin{frame}[fragile,label=sec-2-3-3]{Definición del método para \texttt{task3} con \texttt{UseMethod}}
 \lstset{language=R,numbers=none}
\begin{lstlisting}
myFun.task3 <- function(x, number,...){
  if (!missing(number)) cat('Task no.', number,':\n')
  cat('What: ', x$what,
      '- When:', as.character(x$when),
      '- Priority:', x$priority,
      '\n')
}
\end{lstlisting}

\lstset{language=R,numbers=none}
\begin{lstlisting}
myFun(task1)
\end{lstlisting}

\begin{verbatim}
What:  Write an email - When: 2013-01-01 - Priority: Low
\end{verbatim}

\lstset{language=R,numbers=none}
\begin{lstlisting}
methods(myFun)
\end{lstlisting}

\begin{verbatim}
[1] myFun.default myFun.task3
\end{verbatim}

\lstset{language=R,numbers=none}
\begin{lstlisting}
methods(class='task3')
\end{lstlisting}

\begin{verbatim}
[1] myFun.task3 print.task3
\end{verbatim}
\end{frame}
\section{Clases y métodos S4}
\label{sec-3}

\subsection{Clases en \texttt{S4}}
\label{sec-3-1}
\begin{frame}[fragile,label=sec-3-1-1]{Clases en \texttt{S4}}
 \begin{itemize}
\item Se construyen con \texttt{setClass}, que acepta varios argumentos
\begin{itemize}
\item \texttt{Class}: nombre de la clase.
\item \texttt{representation}: una lista con las clases de cada componente. Los nombres de este vector corresponden a los nombres de los componentes (\texttt{slot}).
\item \texttt{contains}: un vector con las clases que esta nueva clase extiende.
\item \texttt{prototype}: un objeto proporcionando el contenido por defecto para los componentes definidos en \texttt{representation}.
\item \texttt{validity}: a función que comprueba la validez de la clase creada con la información suministrada.
\end{itemize}
\item Una vez que la clase ha sido definida con \texttt{setClass}, se puede crear
un objeto nuevo con \texttt{new}.
\end{itemize}
\end{frame}
\begin{frame}[fragile,label=sec-3-1-2]{Definición de una nueva clase}
 \lstset{language=R,numbers=none}
\begin{lstlisting}
setClass('task',
	 representation=list(what='character',
	   when='Date',
	   priority='character')
	 )
\end{lstlisting}

\lstset{language=R,numbers=none}
\begin{lstlisting}
getClass('task')
\end{lstlisting}

\begin{verbatim}
Class "task" [in ".GlobalEnv"]

Slots:
                                    
Name:       what      when  priority
Class: character      Date character
\end{verbatim}

\lstset{language=R,numbers=none}
\begin{lstlisting}
getSlots('task')
\end{lstlisting}

\begin{verbatim}
       what        when    priority 
"character"      "Date" "character"
\end{verbatim}

\lstset{language=R,numbers=none}
\begin{lstlisting}
slotNames('task')
\end{lstlisting}

\begin{verbatim}
[1] "what"     "when"     "priority"
\end{verbatim}
\end{frame}
\begin{frame}[fragile,label=sec-3-1-3]{Creación de un objeto con la clase definida: \texttt{new}}
 \lstset{language=R,numbers=none}
\begin{lstlisting}
task1 <- new('task', what='Find and fix bugs',
		when=as.Date('2013-03-15'),
		priority='High')
\end{lstlisting}

\lstset{language=R,numbers=none}
\begin{lstlisting}
task1
\end{lstlisting}

\begin{verbatim}
An object of class "task"
Slot "what":
[1] "Find and fix bugs"

Slot "when":
[1] "2013-03-15"

Slot "priority":
[1] "High"
\end{verbatim}
\end{frame}
\begin{frame}[fragile,label=sec-3-1-4]{Funciones para crear objetos}
 \begin{itemize}
\item Es habitual definir funciones que construyen y modifican objetos
para evitar el uso de \texttt{new}:
\end{itemize}
\lstset{language=R,numbers=none}
\begin{lstlisting}
createTask <- function(what, when, priority){
  new('task', what=what, when=when, priority=priority)
  }
\end{lstlisting}

\lstset{language=R,numbers=none}
\begin{lstlisting}
task2 <-createTask(what='Write an email',
		when=as.Date('2013-01-01'),
		priority='Low')
\end{lstlisting}

\lstset{language=R,numbers=none}
\begin{lstlisting}
createTask('Oops', 'Hoy', 3)
\end{lstlisting}

\begin{verbatim}
Error en validObject(.Object) (from #2) : 
  invalid class “task” object: 1: invalid object for slot "when" in class "task": got class "character", should be or extend class "Date"
invalid class “task” object: 2: invalid object for slot "priority" in class "task": got class "numeric", should be or extend class "character"
\end{verbatim}
\end{frame}
\begin{frame}[fragile,label=sec-3-1-5]{Definición de la clase \texttt{ToDo}}
 \lstset{language=R,numbers=none}
\begin{lstlisting}
setClass('ToDo',
	 representation=list(tasks='list')
	 )
\end{lstlisting}

\lstset{language=R,numbers=none}
\begin{lstlisting}
myList <- new('ToDo',
	      tasks=list(t1=task1, t2=task2))
\end{lstlisting}
\end{frame}
\begin{frame}[fragile,label=sec-3-1-6]{Acceso a los slots}
 \begin{itemize}
\item Para extraer información de los \emph{slots} hay que emplear \texttt{@} (a
diferencia de \texttt{\$} en listas y \texttt{data.frame})
\end{itemize}
\lstset{language=R,numbers=none}
\begin{lstlisting}
myList@tasks
\end{lstlisting}

\begin{verbatim}
$t1
An object of class "task"
Slot "what":
[1] "Find and fix bugs"

Slot "when":
[1] "2013-03-15"

Slot "priority":
[1] "High"


$t2
An object of class "task"
Slot "what":
[1] "Write an email"

Slot "when":
[1] "2013-01-01"

Slot "priority":
[1] "Low"
\end{verbatim}
\end{frame}
\begin{frame}[fragile,label=sec-3-1-7]{Acceso a los slots}
 \begin{itemize}
\item El \emph{slot} \texttt{tasks} es una lista: empleamos \texttt{\$} para acceder a sus elementos
\end{itemize}
\lstset{language=R,numbers=none}
\begin{lstlisting}
myList@tasks$t1
\end{lstlisting}

\begin{verbatim}
An object of class "task"
Slot "what":
[1] "Find and fix bugs"

Slot "when":
[1] "2013-03-15"

Slot "priority":
[1] "High"
\end{verbatim}

\begin{itemize}
\item Cada elemento de \texttt{tasks} es un objeto de clase \texttt{task}: empleamos
  \texttt{@} para extraer sus \emph{slots}.
\end{itemize}
\lstset{language=R,numbers=none}
\begin{lstlisting}
myList@tasks$t1@what
\end{lstlisting}

\begin{verbatim}
[1] "Find and fix bugs"
\end{verbatim}
\end{frame}
\begin{frame}[fragile,label=sec-3-1-8]{Problema con los slots definidos como \texttt{list}}
 \begin{itemize}
\item Dado que el slot \texttt{tasks} es una \texttt{list}, podemos añadir cualquier
cosa.
\end{itemize}
\lstset{language=R,numbers=none}
\begin{lstlisting}
myListOops <- new('ToDo',
		  tasks=list(t1='Tarea1',
		    task1, task2))
\end{lstlisting}
\end{frame}
\begin{frame}[fragile,label=sec-3-1-9]{Validación}
 \begin{itemize}
\item Para obligar a que sus elementos sean de clase \texttt{task} debemos añadir
una función de validación.
\end{itemize}
\lstset{language=R,numbers=none}
\begin{lstlisting}
valida <- function (object) {
  if (any(sapply(object@tasks, function(x) !is(x, "task")))) 
    stop("not a list of task objects")
  return(TRUE)
}

setClass('ToDo',
	 representation=list(tasks='list'),
	 validity=valida
	 )
\end{lstlisting}

\lstset{language=R,numbers=none}
\begin{lstlisting}
myListOops <- new('ToDo',
		  tasks=list(t1='Tarea1',
		    task1, task2))
\end{lstlisting}

\begin{verbatim}
Error en validityMethod(object) (from #2) : not a list of task objects
\end{verbatim}
\end{frame}
\begin{frame}[fragile,label=sec-3-1-10]{Funciones para crear objetos}
 \lstset{language=R,numbers=none}
\begin{lstlisting}
createToDo <- function(){
  new('ToDo')
  }
\end{lstlisting}

\lstset{language=R,numbers=none}
\begin{lstlisting}
addTask <- function(object, task){
    ## La siguiente comprobación sólo es necesaria si la
    ## definición de la clase *no* incorpora una función 
    ## validity
    stopifnot(is(task,'task'))
    object@tasks <- c(object@tasks, task)
    object
  }
\end{lstlisting}
\end{frame}

\subsection{Métodos en \texttt{S4}}
\label{sec-3-2}

\begin{frame}[fragile,label=sec-3-2-1]{Métodos en \texttt{S4} \\ \texttt{setMethod}}
 \begin{itemize}
\item Normalmente se definen con \texttt{setMethod}.
\item Hay que definir:
\begin{itemize}
\item la \texttt{signature} (clase de los argumentos para \emph{esta} definición del
método)
\item la función a ejecutar (\texttt{definition}).
\end{itemize}
\item Es necesario que exista un método genérico ya definido. Si no
existe, se define con \texttt{setGeneric}.
\end{itemize}
\lstset{language=R,numbers=none}
\begin{lstlisting}
isGeneric('print')
\end{lstlisting}

\begin{verbatim}
[1] FALSE
\end{verbatim}

\lstset{language=R,numbers=none}
\begin{lstlisting}
setGeneric('print')
\end{lstlisting}

\begin{verbatim}
[1] "print"
\end{verbatim}
\end{frame}
\begin{frame}[fragile,label=sec-3-2-2]{Métodos en \texttt{S4} \\ \texttt{setGeneric} y \texttt{getGeneric}}
 \begin{itemize}
\item Si ya existe un método genérico, la función \texttt{definition} debe tener
todos los argumentos de la función genérica y en el mismo orden.
\end{itemize}
\lstset{language=R,numbers=none}
\begin{lstlisting}
getGeneric('print')
\end{lstlisting}

\begin{verbatim}
standardGeneric for "print" defined from package "base"

function (x, ...) 
standardGeneric("print")
<environment: 0x8f773d4>
Methods may be defined for arguments: x
Use  showMethods("print")  for currently available ones.
\end{verbatim}
\end{frame}
\begin{frame}[fragile,label=sec-3-2-3]{Definición de un método \texttt{print} para \texttt{task}}
 \lstset{language=R,numbers=none}
\begin{lstlisting}
setMethod('print', signature='task',
	  definition=function(x,...){
	    cat('What: ', x@what,
		'- When:', as.character(x@when),
		'- Priority:', x@priority,
		'\n')
	  })
\end{lstlisting}

\begin{verbatim}
[1] "print"
\end{verbatim}


\lstset{language=R,numbers=none}
\begin{lstlisting}
print(task1)
\end{lstlisting}

\begin{verbatim}
What:  Find and fix bugs - When: 2013-03-15 - Priority: High
\end{verbatim}
\end{frame}
\begin{frame}[fragile,label=sec-3-2-4]{Definición de un método \texttt{print} para \texttt{task}}
 \lstset{language=R,numbers=none}
\begin{lstlisting}
setMethod('print', signature='ToDo',
	    definition=function(x,...){
	      cat('This is my ToDo list:\n')
	      tasksList <- x@tasks
	      for (i in seq_along(tasksList)) {
		cat('No.', i, ':')
		print(tasksList[[i]])
		}
	      cat('--------------------\n')
	    })
\end{lstlisting}

\begin{verbatim}
[1] "print"
\end{verbatim}

\lstset{language=R,numbers=none}
\begin{lstlisting}
print(myList)
\end{lstlisting}

\begin{verbatim}
This is my ToDo list:
No. 1 :What:  Find and fix bugs - When: 2013-03-15 - Priority: High 
No. 2 :What:  Write an email - When: 2013-01-01 - Priority: Low 
--------------------
\end{verbatim}
\end{frame}
\subsection{Clases \texttt{S3} con clases y métodos \texttt{S4}}
\label{sec-3-3}

\begin{frame}[fragile,label=sec-3-3-1]{Clases \texttt{S3} con clases y métodos \texttt{S4}}
 \begin{itemize}
\item Para usar objetos de clase \texttt{S3} en \texttt{signatures} de métodos \texttt{S4} o
como contenido de \texttt{slots} de una clase \texttt{S4} hay que registrarlos con
\texttt{setOldClass}:
\end{itemize}
\lstset{language=R,numbers=none}
\begin{lstlisting}
setOldClass('lm')
\end{lstlisting}

\lstset{language=R,numbers=none}
\begin{lstlisting}
getClass('lm')
\end{lstlisting}

\begin{verbatim}
Virtual Class "lm" [package "methods"]

Slots:
                
Name:   .S3Class
Class: character

Extends: "oldClass"

Known Subclasses: 
Class "mlm", directly
Class "aov", directly
Class "glm", directly
Class "maov", by class "mlm", distance 2
Class "glm.null", by class "glm", distance 2
\end{verbatim}
\end{frame}
\begin{frame}[fragile,label=sec-3-3-2]{Ejemplo con \texttt{lm} y \texttt{xyplot}}
 \begin{itemize}
\item Definimos un método genérico para \texttt{xyplot}
\end{itemize}
\lstset{language=R,numbers=none}
\begin{lstlisting}
library(lattice)
setGeneric('xyplot')
\end{lstlisting}

\begin{verbatim}
[1] "xyplot"
\end{verbatim}

\begin{itemize}
\item Definimos un método para la clase \texttt{lm} usando \texttt{xyplot}.
\end{itemize}
\lstset{language=R,numbers=none}
\begin{lstlisting}
setMethod('xyplot',
	  signature=c(x='lm', data='missing'),
	  definition=function(x, data,
	    ...){
	    fitted <- fitted(x)
	    residuals <- residuals(x)
	    xyplot(residuals ~ fitted,...)
	    })
\end{lstlisting}

\begin{verbatim}
[1] "xyplot"
\end{verbatim}
\end{frame}
\begin{frame}[fragile,label=sec-3-3-3]{Ejemplo con \texttt{lm} y \texttt{xyplot}}
 \begin{itemize}
\item Recuperamos la regresión que empleamos en el apartado de Estadística:
\end{itemize}
\lstset{language=R,numbers=none}
\begin{lstlisting}
lmFertEdu <- lm(Fertility ~ Education, data = swiss)
\end{lstlisting}

\lstset{language=R,numbers=none}
\begin{lstlisting}
xyplot(lmFertEdu, col='red', pch=19, type=c('p', 'g'))
\end{lstlisting}

\includegraphics[width=.9\linewidth]{/home/oscar/github/intro/xyplotS4.pdf}
\end{frame}
% Emacs 24.3.1 (Org mode 8.2.1)
\end{document}