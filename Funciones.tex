% Created 2013-03-02 sáb 18:01
\documentclass[xcolor={usenames,svgnames,dvipsnames}]{beamer}
\usepackage[utf8]{inputenc}
\usepackage[T1]{fontenc}
\usepackage{fixltx2e}
\usepackage{graphicx}
\usepackage{longtable}
\usepackage{float}
\usepackage{wrapfig}
\usepackage{soul}
\usepackage{textcomp}
\usepackage{marvosym}
\usepackage{wasysym}
\usepackage{latexsym}
\usepackage{amssymb}
\usepackage{hyperref}
\tolerance=1000
\usepackage{color}
\usepackage{listings}
\AtBeginSection[]{\begin{frame}<beamer>\frametitle{Contenidos}\tableofcontents[currentsection]\end{frame}}
\lstset{keywordstyle=\color{blue}, commentstyle=\color{gray!90}, basicstyle=\ttfamily\footnotesize, columns=fullflexible, breaklines=false,linewidth=\textwidth, backgroundcolor=\color{gray!23}, basewidth={0.5em,0.4em}, literate={á}{{\'a}}1 {ñ}{{\~n}}1 {é}{{\'e}}1 {ó}{{\'o}}1 {º}{{\textordmasculine}}1}
\usepackage{mathpazo}
\setbeamercovered{transparent}
\usefonttheme{serif} 
\usetheme{Goettingen}
\hypersetup{colorlinks=true, linkcolor=Blue, urlcolor=Blue}
\usepackage{fancyvrb}
\DefineVerbatimEnvironment{verbatim}{Verbatim}{fontsize=\tiny, formatcom = {\color{black!70}}}
\providecommand{\alert}[1]{\textbf{#1}}

\title{Funciones}
\author{Oscar Perpiñán Lamigueiro}
\date{19 de Febrero de 2013}
\hypersetup{
  pdfkeywords={},
  pdfsubject={},
  pdfcreator={Emacs Org-mode version 7.8.11}}

\begin{document}

\maketitle




\section{Conceptos Básicos}
\label{sec-1}
\begin{frame}
\frametitle{Fuentes de información}
\label{sec-1-1}

\begin{itemize}
\item \href{http://cran.r-project.org/doc/manuals/R-intro.html}{R introduction}
\item \href{http://cran.r-project.org/doc/manuals/R-lang.html}{R Language Definition}
\item \href{http://books.google.es/books/about/Software_for_Data_Analysis.html}{Software for Data Analysis}
\end{itemize}
\end{frame}
\begin{frame}
\frametitle{Componentes de una función}
\label{sec-1-2}

\begin{itemize}
\item Una función se define con \texttt{function}
\end{itemize}
\begin{center}
\texttt{name <- function(arg\_1, arg\_2, ...) expression}
\end{center}

\begin{itemize}
\item Está compuesta por:
\begin{itemize}
\item Nombre de la función (\texttt{name})
\item Argumentos (\texttt{arg\_1}, \texttt{arg\_2}, \texttt{...})
\item Cuerpo (\texttt{expression}): emplea los argumentos para generar un resultado
\end{itemize}
\end{itemize}
\end{frame}
\begin{frame}[fragile]
\frametitle{Mi primera función}
\label{sec-1-3}

\begin{itemize}
\item Definición
\end{itemize}

\lstset{language=R}
\begin{lstlisting}
myFun <- function(x, y){
    x + y
    }
\end{lstlisting}


\begin{itemize}
\item Argumentos
\end{itemize}

\lstset{language=R}
\begin{lstlisting}
formals(myFun)
\end{lstlisting}

\begin{verbatim}
 $x
 
 
 $y
\end{verbatim}

\begin{itemize}
\item Cuerpo
\end{itemize}

\lstset{language=R}
\begin{lstlisting}
body(myFun)
\end{lstlisting}

\begin{verbatim}
 {
     x + y
 }
\end{verbatim}
\end{frame}
\begin{frame}[fragile]
\frametitle{Mi primera función}
\label{sec-1-4}


\lstset{language=R}
\begin{lstlisting}
myFun(1, 2)
\end{lstlisting}

\begin{verbatim}
 [1] 3
\end{verbatim}


\lstset{language=R}
\begin{lstlisting}
myFun(1:10, 21:30)
\end{lstlisting}

\begin{verbatim}
  [1] 22 24 26 28 30 32 34 36 38 40
\end{verbatim}


\lstset{language=R}
\begin{lstlisting}
myFun(1:10, 3)
\end{lstlisting}

\begin{verbatim}
  [1]  4  5  6  7  8  9 10 11 12 13
\end{verbatim}
\end{frame}
\begin{frame}[fragile]
\frametitle{Argumentos: nombre y orden}
\label{sec-1-5}


\begin{itemize}
\item Una función identifica sus argumentos por su nombre y por su orden (sin nombre)
\end{itemize}


\lstset{language=R}
\begin{lstlisting}
power <- function(x, exp){
    x^exp
    }
\end{lstlisting}



\lstset{language=R}
\begin{lstlisting}
power(x=1:10, exp=2)
\end{lstlisting}

\begin{verbatim}
  [1]   1   4   9  16  25  36  49  64  81 100
\end{verbatim}


\lstset{language=R}
\begin{lstlisting}
power(1:10, exp=2)
\end{lstlisting}

\begin{verbatim}
  [1]   1   4   9  16  25  36  49  64  81 100
\end{verbatim}


\lstset{language=R}
\begin{lstlisting}
power(exp=2, x=1:10)
\end{lstlisting}

\begin{verbatim}
  [1]   1   4   9  16  25  36  49  64  81 100
\end{verbatim}
\end{frame}
\begin{frame}[fragile]
\frametitle{Argumentos: valores por defecto}
\label{sec-1-6}

\begin{itemize}
\item Se puede asignar un valor por defecto a los argumentos
\end{itemize}

\lstset{language=R}
\begin{lstlisting}
power <- function(x, exp=2){
    x ^ exp
    }
\end{lstlisting}



\lstset{language=R}
\begin{lstlisting}
power(1:10)
\end{lstlisting}

\begin{verbatim}
  [1]   1   4   9  16  25  36  49  64  81 100
\end{verbatim}


\lstset{language=R}
\begin{lstlisting}
power(1:10, 2)
\end{lstlisting}

\begin{verbatim}
  [1]   1   4   9  16  25  36  49  64  81 100
\end{verbatim}
\end{frame}
\begin{frame}[fragile]
\frametitle{Funciones sin argumentos}
\label{sec-1-7}


\lstset{language=R}
\begin{lstlisting}
hello <- function(){
    print('Hello world!')
    }
\end{lstlisting}



\lstset{language=R}
\begin{lstlisting}
hello()
\end{lstlisting}

\begin{verbatim}
 [1] "Hello world!"
\end{verbatim}
\end{frame}
\begin{frame}[fragile]
\frametitle{Argumentos sin nombre: \texttt{...}}
\label{sec-1-8}


\lstset{language=R}
\begin{lstlisting}
pwrSum <- function(x, power, ...){
    sum(x ^ power, ...)
    }
\end{lstlisting}



\lstset{language=R}
\begin{lstlisting}
x <- 1:10
pwrSum(x, 2)
\end{lstlisting}

\begin{verbatim}
 [1] 385
\end{verbatim}


\lstset{language=R}
\begin{lstlisting}
x <- c(1:5, NA, 6:9, NA, 10)
pwrSum(x, 2)
\end{lstlisting}

\begin{verbatim}
 [1] NA
\end{verbatim}


\lstset{language=R}
\begin{lstlisting}
pwrSum(x, 2, na.rm=TRUE)
\end{lstlisting}

\begin{verbatim}
 [1] 385
\end{verbatim}
\end{frame}
\begin{frame}[fragile]
\frametitle{Argumentos ausentes: \texttt{missing}}
\label{sec-1-9}


\lstset{language=R}
\begin{lstlisting}
suma10 <- function(x, y){
    if (missing(y)) y <- 10
    x + y
    }
\end{lstlisting}



\lstset{language=R}
\begin{lstlisting}
suma10(1:10)
\end{lstlisting}

\begin{verbatim}
  [1] 11 12 13 14 15 16 17 18 19 20
\end{verbatim}
\end{frame}
\begin{frame}[fragile]
\frametitle{Control de errores: \texttt{stopifnot}}
\label{sec-1-10}


\lstset{language=R}
\begin{lstlisting}
foo <- function(x, y){
    stopifnot(is.numeric(x) & is.numeric(y))
    x + y
    }
\end{lstlisting}



\lstset{language=R}
\begin{lstlisting}
foo(1:10, 21:30)
\end{lstlisting}

\begin{verbatim}
  [1] 22 24 26 28 30 32 34 36 38 40
\end{verbatim}


\lstset{language=R}
\begin{lstlisting}
foo(1:10, 'a')
\end{lstlisting}

\begin{verbatim}
 Error: is.numeric(x) & is.numeric(y) is not TRUE
\end{verbatim}
\end{frame}
\begin{frame}[fragile]
\frametitle{Control de errores: \texttt{stop}}
\label{sec-1-11}


\lstset{language=R}
\begin{lstlisting}
foo <- function(x, y){
    if (!(is.numeric(x) & is.numeric(y))){
        stop('arguments must be numeric.')
        } else { x + y }
    }
\end{lstlisting}



\lstset{language=R}
\begin{lstlisting}
foo(2, 3)
\end{lstlisting}

\begin{verbatim}
 [1] 5
\end{verbatim}


\lstset{language=R}
\begin{lstlisting}
foo(2, 'a')
\end{lstlisting}

\begin{verbatim}
 Error en foo(2, "a") : arguments must be numeric.
\end{verbatim}
\end{frame}
\section{Lexical scope}
\label{sec-2}
\begin{frame}[fragile]
\frametitle{Clases de variables}
\label{sec-2-1}

\begin{itemize}
\item Las variables que se emplean en el cuerpo de una función pueden
  dividirse en:
\begin{itemize}
\item Parámetros formales (argumentos): \texttt{x}, \texttt{y}
\item Variables locales (definiciones internas): \texttt{z}, \texttt{w}, \texttt{m}
\item Variables libres: \texttt{a}, \texttt{b}
\end{itemize}
\end{itemize}

\lstset{language=R}
\begin{lstlisting}
myFun <- function(x, y){
    z <- x^2
    w <- y^3
    m <- a*z + b*w
    m
    }
\end{lstlisting}



\lstset{language=R}
\begin{lstlisting}
a <- 10
b <- 20
myFun(2, 3)
\end{lstlisting}

\begin{verbatim}
 [1] 580
\end{verbatim}
\end{frame}
\begin{frame}[fragile]
\frametitle{Lexical scope}
\label{sec-2-2}


\begin{itemize}
\item Las variables libres deben estar disponibles en el entorno
  (\texttt{environment}) en el que la función ha sido creada.
\end{itemize}

\lstset{language=R}
\begin{lstlisting}
environment(myFun)
\end{lstlisting}

\begin{verbatim}
 <environment: R_GlobalEnv>
\end{verbatim}


\lstset{language=R}
\begin{lstlisting}
ls()
\end{lstlisting}

\begin{verbatim}
  [1] "a"           "add"         "anidada"     "b"           "constructor"
  [6] "fib"         "foo"         "frac"        "hello"       "lista"      
 [11] "ll"          "M"           "myFoo"       "myFooenv"    "myFun"      
 [16] "noise"       "power"       "pwrSum"      "ruido"       "suma10"     
 [21] "sumNoise"    "sumProd"     "sumSq"       "tmp"         "vals"       
 [26] "x"           "zz"
\end{verbatim}
\end{frame}
\begin{frame}[fragile]
\frametitle{Lexical scope: funciones anidadas}
\label{sec-2-3}


\lstset{language=R}
\begin{lstlisting}
anidada <- function(x, y){
    xn <- 2
    yn <- 3
    interna <- function(x, y){
        sum(x^xn, y^yn)
        }
    print(environment(interna))
    interna(x, y)
    }
\end{lstlisting}



\lstset{language=R}
\begin{lstlisting}
anidada(1:3, 2:4)
\end{lstlisting}

\begin{verbatim}
 <environment: 0xa645674>
 [1] 113
\end{verbatim}


\lstset{language=R}
\begin{lstlisting}
sum((1:3)^2, (2:4)^3)
\end{lstlisting}

\begin{verbatim}
 [1] 113
\end{verbatim}
\end{frame}
\begin{frame}[fragile]
\frametitle{Lexical scope: funciones anidadas}
\label{sec-2-4}


\lstset{language=R}
\begin{lstlisting}
xn
\end{lstlisting}

\begin{verbatim}
 Error: objeto 'xn' no encontrado
\end{verbatim}


\lstset{language=R}
\begin{lstlisting}
yn
\end{lstlisting}

\begin{verbatim}
 Error: objeto 'yn' no encontrado
\end{verbatim}


\lstset{language=R}
\begin{lstlisting}
interna
\end{lstlisting}

\begin{verbatim}
 Error: objeto 'interna' no encontrado
\end{verbatim}
\end{frame}
\begin{frame}[fragile]
\frametitle{Funciones que devuelven funciones}
\label{sec-2-5}


\lstset{language=R}
\begin{lstlisting}
constructor <- function(m, n){
    function(x){
        m*x + n
        }
    }
\end{lstlisting}



\lstset{language=R}
\begin{lstlisting}
myFoo <- constructor(10, 3)
myFoo
\end{lstlisting}

\begin{verbatim}
 function(x){
         m*x + n
         }
 <environment: 0xa63e3b8>
\end{verbatim}
\end{frame}
\begin{frame}[fragile]
\frametitle{Funciones que devuelven funciones}
\label{sec-2-6}


\lstset{language=R}
\begin{lstlisting}
class(myFoo)
\end{lstlisting}

\begin{verbatim}
 [1] "function"
\end{verbatim}


\lstset{language=R}
\begin{lstlisting}
environment(myFoo)
\end{lstlisting}

\begin{verbatim}
 <environment: 0xa63e3b8>
\end{verbatim}


\lstset{language=R}
\begin{lstlisting}
ls()
\end{lstlisting}

\begin{verbatim}
  [1] "a"           "add"         "anidada"     "b"           "constructor"
  [6] "fib"         "foo"         "frac"        "hello"       "lista"      
 [11] "ll"          "M"           "myFoo"       "myFooenv"    "myFun"      
 [16] "noise"       "power"       "pwrSum"      "ruido"       "suma10"     
 [21] "sumNoise"    "sumProd"     "sumSq"       "tmp"         "vals"       
 [26] "x"           "zz"
\end{verbatim}


\lstset{language=R}
\begin{lstlisting}
ls(env=environment(myFoo))
\end{lstlisting}

\begin{verbatim}
 [1] "m" "n"
\end{verbatim}


\lstset{language=R}
\begin{lstlisting}
get('m', env=environment(myFoo))
\end{lstlisting}

\begin{verbatim}
 [1] 10
\end{verbatim}


\lstset{language=R}
\begin{lstlisting}
get('n', env=environment(myFoo))
\end{lstlisting}

\begin{verbatim}
 [1] 3
\end{verbatim}
\end{frame}
\section{Debug y profiling}
\label{sec-3}
\begin{frame}[fragile]
\frametitle{\texttt{traceback}}
\label{sec-3-1}


\lstset{language=R}
\begin{lstlisting}
sumSq <- function(x, ...){
    sum(x ^ 2, ...)
    }

sumProd <- function(x, y, ...){
    xs <- sumSq(x, ...)
    ys <- sumSq(y, ...)
    xs * ys
    }
\end{lstlisting}



\lstset{language=R}
\begin{lstlisting}
sumProd(rnorm(10), runif(10))
\end{lstlisting}

\begin{verbatim}
 [1] 15.21856
\end{verbatim}


\lstset{language=R}
\begin{lstlisting}
sumProd(rnorm(10), letters[1:10])
\end{lstlisting}

\begin{verbatim}
 Error en x^2 : argumento no-numérico para operador binario
\end{verbatim}


\lstset{language=R}
\begin{lstlisting}
traceback()
\end{lstlisting}

\begin{verbatim}
 3: x^2 at #2
 2: sumSq(y, ...) at #3
 1: sumProd(rnorm(10), letters[1:10])
\end{verbatim}
\end{frame}
\begin{frame}
\frametitle{Debugger}
\label{sec-3-2}
\end{frame}
\begin{frame}[fragile]
\frametitle{\texttt{system.time}}
\label{sec-3-3}


\lstset{language=R}
\begin{lstlisting}
noise <- function(sd)rnorm(1000, mean=0, sd=sd)
\end{lstlisting}



\lstset{language=R}
\begin{lstlisting}
sumNoise <- function(nComponents){
    vals <- sapply(seq_len(nComponents), noise)
    rowSums(vals)
    }
\end{lstlisting}



\lstset{language=R}
\begin{lstlisting}
system.time(sumNoise(1000))
\end{lstlisting}

\begin{verbatim}
    user  system elapsed 
   0.244   0.020   0.265
\end{verbatim}
\end{frame}
\begin{frame}[fragile]
\frametitle{\texttt{Rprof}}
\label{sec-3-4}

\begin{itemize}
\item Usaremos un fichero temporal
\end{itemize}

\lstset{language=R}
\begin{lstlisting}
tmp <- tempfile()
\end{lstlisting}


\begin{itemize}
\item Activamos la toma de información
\end{itemize}

\lstset{language=R}
\begin{lstlisting}
Rprof(tmp)
\end{lstlisting}


\begin{itemize}
\item Ejecutamos el código a analizar
\end{itemize}

\lstset{language=R}
\begin{lstlisting}
zz <- sumNoise(1000)
\end{lstlisting}
\end{frame}
\begin{frame}[fragile]
\frametitle{\texttt{Rprof}}
\label{sec-3-5}

\begin{itemize}
\item Paramos el análisis
\end{itemize}

\lstset{language=R}
\begin{lstlisting}
Rprof()
\end{lstlisting}


\begin{itemize}
\item Extraemos el resumen
\end{itemize}

\lstset{language=R}
\begin{lstlisting}
summaryRprof(tmp)
\end{lstlisting}


\begin{verbatim}
$by.self
        self.time self.pct total.time total.pct
"rnorm"      0.24       75       0.24        75
"array"      0.08       25       0.08        25

$by.total
                 total.time total.pct self.time self.pct
"sapply"               0.32       100      0.00        0
"sumNoise"             0.32       100      0.00        0
"rnorm"                0.24        75      0.24       75
"FUN"                  0.24        75      0.00        0
"lapply"               0.24        75      0.00        0
"array"                0.08        25      0.08       25
"simplify2array"       0.08        25      0.00        0

$sample.interval
[1] 0.02

$sampling.time
[1] 0.32
\end{verbatim}
\end{frame}
\section{Sofisticaciones}
\label{sec-4}
\begin{frame}[fragile]
\frametitle{\texttt{do.call}}
\label{sec-4-1}

\begin{itemize}
\item Ejemplo: sumar los componentes de una lista
\end{itemize}

\lstset{language=R}
\begin{lstlisting}
lista <- list(a=rnorm(100), b=runif(100), c=rexp(100))
with(lista, sum(a + b + c))
\end{lstlisting}

\begin{verbatim}
 [1] 167.6913
\end{verbatim}

\begin{itemize}
\item En lugar de nombrar los componentes, creamos una llamada a una
  función con \texttt{do.call}
\end{itemize}

\lstset{language=R}
\begin{lstlisting}
do.call(sum, lista)
\end{lstlisting}

\begin{verbatim}
 [1] 167.6913
\end{verbatim}
\end{frame}
\begin{frame}[fragile]
\frametitle{\texttt{do.call}}
\label{sec-4-2}


\begin{itemize}
\item Se emplea frecuentemente con el resultado de \texttt{lapply}
\end{itemize}

\lstset{language=R}
\begin{lstlisting}
x <- rnorm(5)
ll <- lapply(1:5, function(i)x^i)
do.call(rbind, ll)
\end{lstlisting}

\begin{verbatim}
           [,1]         [,2]      [,3]      [,4]        [,5]
 [1,]  2.477327 -0.311525664 -1.342538  1.650633 0.365936506
 [2,]  6.137148  0.097048239  1.802409  2.724589 0.133909527
 [3,] 15.203720 -0.030233017 -2.419804  4.497296 0.049002384
 [4,] 37.664581  0.009418361  3.248680  7.423385 0.017931761
 [5,] 93.307473 -0.002934061 -4.361478 12.253284 0.006561886
\end{verbatim}

\begin{itemize}
\item Este mismo ejemplo puede resolverse con \texttt{sapply}
\end{itemize}

\lstset{language=R}
\begin{lstlisting}
sapply(1:5, function(i)x^i)
\end{lstlisting}

\begin{verbatim}
            [,1]       [,2]        [,3]         [,4]         [,5]
 [1,]  2.4773267 6.13714765 15.20371982 37.664581254 93.307473313
 [2,] -0.3115257 0.09704824 -0.03023302  0.009418361 -0.002934061
 [3,] -1.3425384 1.80240946 -2.41980398  3.248679858 -4.361477583
 [4,]  1.6506329 2.72458904  4.49729637  7.423385446 12.253284406
 [5,]  0.3659365 0.13390953  0.04900238  0.017931761  0.006561886
\end{verbatim}
\end{frame}
\begin{frame}[fragile]
\frametitle{\texttt{Reduce}}
\label{sec-4-3}

\begin{itemize}
\item Combina sucesivamente los elementos de un objeto aplicando una
  función binaria
\end{itemize}

\lstset{language=R}
\begin{lstlisting}
Reduce('+', 1:10)
## equivalente a 
## sum(1:10)
\end{lstlisting}

\begin{verbatim}
 [1] 55
\end{verbatim}
\end{frame}
\begin{frame}[fragile]
\frametitle{\texttt{Reduce}}
\label{sec-4-4}


\lstset{language=R}
\begin{lstlisting}
Reduce('/', 1:10)
\end{lstlisting}

\begin{verbatim}
 [1] 2.755732e-07
\end{verbatim}


\lstset{language=R}
\begin{lstlisting}
Reduce(paste, LETTERS[1:5])
\end{lstlisting}

\begin{verbatim}
 [1] "A B C D E"
\end{verbatim}


\lstset{language=R}
\begin{lstlisting}
foo <- function(u, v)u + 1 /v
Reduce(foo, c(3, 7, 15, 1, 292), right=TRUE)
## equivalente a
## foo(3, foo(7, foo(15, foo(1, 292))))
\end{lstlisting}

\begin{verbatim}
 [1] 3.141593
\end{verbatim}
\end{frame}
\begin{frame}[fragile]
\frametitle{Funciones recursivas}
\label{sec-4-5}

\begin{itemize}
\item \href{http://en.wikibooks.org/wiki/R_Programming/Working_with_functions#Functions_as_Objects}{Serie de Fibonnaci}
\end{itemize}

\lstset{language=R}
\begin{lstlisting}
fib <- function(n){
    if (n>2) {
        c(fib(n-1),
          sum(tail(fib(n-1),2)))
    } else if (n>=0) rep(1,n)
    }
\end{lstlisting}



\lstset{language=R}
\begin{lstlisting}
fib(10)
\end{lstlisting}

\begin{verbatim}
  [1]  1  1  2  3  5  8 13 21 34 55
\end{verbatim}
\end{frame}

\end{document}