% Created 2022-06-15 mié 20:11
% Intended LaTeX compiler: pdflatex
\documentclass[aspectratio=169, usenames,svgnames,dvipsnames]{beamer}
\usepackage[utf8]{inputenc}
\usepackage[T1]{fontenc}
\usepackage{graphicx}
\usepackage{grffile}
\usepackage{longtable}
\usepackage{wrapfig}
\usepackage{rotating}
\usepackage[normalem]{ulem}
\usepackage{amsmath}
\usepackage{textcomp}
\usepackage{amssymb}
\usepackage{capt-of}
\usepackage{hyperref}
\usepackage{color}
\usepackage{listings}
\usepackage[spanish]{babel}
\setbeamercolor{alerted text}{fg=Blue}
\setbeamerfont{alerted text}{series=\bfseries}
\setbeamercolor{block title}{bg=structure.fg!20!bg!50!bg}
\setbeamercolor{block body}{use=block title,bg=block title.bg}
\AtBeginSubsection[]{\begin{frame}[plain]\tableofcontents[currentsubsection,sectionstyle=show/shaded,subsectionstyle=show/shaded/hide]\end{frame}}
\AtBeginSection[]{\begin{frame}[plain]\tableofcontents[currentsection,hideallsubsections]\end{frame}}
\lstset{keywordstyle=\color{blue}, commentstyle=\color{gray!90}, basicstyle=\ttfamily\small, columns=fullflexible, breaklines=true,linewidth=\textwidth, backgroundcolor=\color{gray!23}, basewidth={0.5em,0.4em}, literate={á}{{\'a}}1 {ñ}{{\~n}}1 {é}{{\'e}}1 {ó}{{\'o}}1 {º}{{\textordmasculine}}1, showstringspaces=false}
\usepackage{mathpazo}
\hypersetup{colorlinks=true, linkcolor=Blue, urlcolor=Blue}
\usepackage{fancyvrb}
\DefineVerbatimEnvironment{verbatim}{Verbatim}{fontsize=\tiny, formatcom = {\color{black!70}}}
\usetheme{Boadilla}
\usecolortheme{rose}
\usefonttheme{serif}
\author{\href{https://oscarperpinan.github.io}{Oscar Perpiñán Lamigueiro}}
\date{}
\title{Gráficos con R}
\institute[UPM]{Universidad Politécnica de Madrid}
\beamertemplatenavigationsymbolsempty
\setbeamertemplate{footline}[frame number]
\setbeamertemplate{itemize items}[triangle]
\setbeamertemplate{enumerate items}[circle]
\setbeamertemplate{section in toc}[circle]
\setbeamertemplate{subsection in toc}[circle]
\hypersetup{
 pdfauthor={\href{https://oscarperpinan.github.io}{Oscar Perpiñán Lamigueiro}},
 pdftitle={Gráficos con R},
 pdfkeywords={},
 pdfsubject={},
 pdfcreator={Emacs 27.1 (Org mode 9.4.6)}, 
 pdflang={Spanish}}
\begin{document}

\maketitle


\section{Introducción}
\label{sec:orgfd0e472}
\begin{frame}[label={sec:org2785367},fragile]{Base y grid}
 \begin{itemize}
\item En \texttt{R} existen dos formas de generar gráficos:
\begin{itemize}
\item Base graphics
\item Grid graphics
\end{itemize}
\item Los gráficos base sólo producen un resultado gráfico, pero no un objeto.
\item Los gráficos \texttt{grid} generan un resultado gráfico \alert{y} un objeto.
\item Dentro del conjunto \texttt{grid} existen dos grandes paquetes: \texttt{lattice} y \texttt{ggplot2}.
\end{itemize}
\end{frame}

\begin{frame}[label={sec:orgf871320},fragile]{Gráficos \texttt{lattice}}
 \begin{itemize}
\item Implementación de los gráficos \emph{trellis}, \emph{The Elements of Graphing Data} de Cleveland)

\item Estructura matricial de paneles definida a través de una fórmula.
\end{itemize}

\lstset{language=r,label= ,caption= ,captionpos=b,numbers=none}
\begin{lstlisting}
library(lattice)

xyplot(wt ~ mpg | am, data = mtcars, groups = cyl)
\end{lstlisting}

\begin{itemize}
\item Documentación: \href{http://lmdvr.r-forge.r-project.org/figures/figures.html}{Código y Figuras del libro}
\end{itemize}
\end{frame}

\begin{frame}[label={sec:org7712b92},fragile]{Gráficos \texttt{ggplot2}}
 \begin{itemize}
\item Implementación de \emph{The Grammar of Graphics} de Wilkinson.

\item Combinación de funciones que proporcionan los componentes (capas) del gráfico.
\end{itemize}

\lstset{language=r,label= ,caption= ,captionpos=b,numbers=none}
\begin{lstlisting}
library(ggplot2)

ggplot(mtcars, aes(mpg, wt)) +
    geom_point(aes(colour=factor(cyl))) +
    facet_grid(. ~ am)
\end{lstlisting}

\begin{itemize}
\item \href{http://docs.ggplot2.org/current/}{Documentación de ggplot2}
\item \href{http://ggplot2.org/book/}{Codigo del libro}
\item \href{http://learnr.wordpress.com/2009/06/28/ggplot2-version-of-figures-in-lattice-multivariate-data-visualization-with-r-part-1/}{ggplot2 desde lattice} (\href{http://learnr.files.wordpress.com/2009/08/latbook.pdf}{PDF})
\end{itemize}
\end{frame}

\section{Datos de ejemplo}
\label{sec:org2baa8f6}
\begin{frame}[label={sec:org9883a3a},fragile]{Leemos desde el archivo local}
 \lstset{language=r,label= ,caption= ,captionpos=b,numbers=none}
\begin{lstlisting}
  aranjuez <- read.csv('data/aranjuez.csv')

  summary(aranjuez)
\end{lstlisting}
\end{frame}

\begin{frame}[label={sec:org154cde7},fragile]{Añadimos algunas columnas}
 \lstset{language=r,label= ,caption= ,captionpos=b,numbers=none}
\begin{lstlisting}
aranjuez$date <- as.Date(aranjuez$X)
\end{lstlisting}
\lstset{language=r,label= ,caption= ,captionpos=b,numbers=none}
\begin{lstlisting}
aranjuez$month <- as.numeric(
    format(aranjuez$date, '%m'))

aranjuez$year <- as.numeric(
    format(aranjuez$date, '%Y'))

aranjuez$day <- as.numeric(
    format(aranjuez$date, '%j'))

aranjuez$quarter <- quarters(aranjuez$date)
      
\end{lstlisting}
\end{frame}

\section{Catálogo de gráficos}
\label{sec:orgcb2f11e}

\begin{frame}[label={sec:orge4ea54a},fragile]{Gráfico de dispersión de puntos}
 \lstset{language=r,label= ,caption= ,captionpos=b,numbers=none}
\begin{lstlisting}
xyplot(Radiation ~ TempAvg, data=aranjuez)
\end{lstlisting}

\lstset{language=r,label= ,caption= ,captionpos=b,numbers=none}
\begin{lstlisting}
ggplot(aranjuez, aes(TempAvg, Radiation)) + 
    geom_point()
\end{lstlisting}
\end{frame}

\begin{frame}[label={sec:orgc4c95b7}]{}
\begin{center}
\includegraphics[height=\textheight]{figs/xyplot.png}
\end{center}
\end{frame}

\begin{frame}[label={sec:org7fd3482},fragile]{Añadimos rejilla}
 \lstset{language=r,label= ,caption= ,captionpos=b,numbers=none}
\begin{lstlisting}
xyplot(Radiation ~ TempAvg, data=aranjuez,
       grid = TRUE)
\end{lstlisting}
\end{frame}

\begin{frame}[label={sec:org18aef03}]{}
\begin{center}
\includegraphics[height=\textheight]{figs/xyplotPG.png}
\end{center}
\end{frame}


\begin{frame}[label={sec:orgd69e362},fragile]{Añadimos regresión lineal}
 \lstset{language=r,label= ,caption= ,captionpos=b,numbers=none}
\begin{lstlisting}
xyplot(Radiation ~ TempAvg, data=aranjuez,
       type=c('p', 'r'), grid = TRUE,
       lwd=2, col.line='black')
  
\end{lstlisting}

\lstset{language=r,label= ,caption= ,captionpos=b,numbers=none}
\begin{lstlisting}
ggplot(aranjuez, aes(TempAvg, Radiation)) + 
    geom_point() +
    geom_smooth(method = "lm")
\end{lstlisting}
\end{frame}

\begin{frame}[label={sec:org1045847}]{}
\begin{center}
\includegraphics[height=\textheight]{figs/xyplotPRG.png}
\end{center}
\end{frame}


\begin{frame}[label={sec:orgf6fe697},fragile]{Añadimos ajuste local}
 \lstset{language=r,label= ,caption= ,captionpos=b,numbers=none}
\begin{lstlisting}
xyplot(Radiation ~ TempAvg, data=aranjuez,
       type=c('p', 'smooth'), grid = TRUE,
       lwd=2, col.line='black')
\end{lstlisting}

\lstset{language=r,label= ,caption= ,captionpos=b,numbers=none}
\begin{lstlisting}
ggplot(aranjuez, aes(TempAvg, Radiation)) + 
    geom_point() +
    geom_smooth()
\end{lstlisting}
\end{frame}

\begin{frame}[label={sec:org75d42d8}]{}
\begin{center}
\includegraphics[height=\textheight]{figs/xyplotSmooth.png}
\end{center}
\end{frame}


\begin{frame}[label={sec:orga3e186b},fragile]{Paneles}
 \lstset{language=r,label= ,caption= ,captionpos=b,numbers=none}
\begin{lstlisting}
xyplot(Radiation ~ TempAvg|factor(year),
       data=aranjuez)
\end{lstlisting}

\lstset{language=r,label= ,caption= ,captionpos=b,numbers=none}
\begin{lstlisting}
ggplot(aranjuez, aes(TempAvg, Radiation)) + 
    geom_point() +
    facet_wrap(~factor(year))
\end{lstlisting}
\end{frame}
\begin{frame}[label={sec:org37f7cef}]{}
\begin{center}
\includegraphics[height=\textheight]{figs/xyplotYear.png}
\end{center}
\end{frame}

\begin{frame}[label={sec:org02400df},fragile]{Grupos}
 \lstset{language=r,label= ,caption= ,captionpos=b,numbers=none}
\begin{lstlisting}
xyplot(Radiation ~ TempAvg, groups=quarter,
       data=aranjuez, auto.key=list(space='right'))
\end{lstlisting}

\lstset{language=r,label= ,caption= ,captionpos=b,numbers=none}
\begin{lstlisting}
ggplot(aranjuez, aes(TempAvg, Radiation,
                     color = quarter)) + 
    geom_point()
\end{lstlisting}
\end{frame}

\begin{frame}[label={sec:orgca32312}]{}
\begin{center}
\includegraphics[height=\textheight]{figs/xyplotQuarter.png}
\end{center}
\end{frame}

\begin{frame}[label={sec:org8df30bd},fragile]{Paneles y grupos}
 \lstset{language=r,label= ,caption= ,captionpos=b,numbers=none}
\begin{lstlisting}
xyplot(Radiation ~ TempAvg|factor(year),
       groups=quarter,
       data=aranjuez,
       layout=c(4, 2),
       auto.key=list(space='right'))
\end{lstlisting}

\lstset{language=r,label= ,caption= ,captionpos=b,numbers=none}
\begin{lstlisting}
ggplot(aranjuez, aes(TempAvg, Radiation,
                     color = quarter)) + 
    geom_point() +
    facet_wrap(~factor(year))
\end{lstlisting}
\end{frame}

\begin{frame}[label={sec:orgc6a17a4}]{}
\begin{center}
\includegraphics[height=\textheight]{figs/xyplotQuarterYear.png}
\end{center}
\end{frame}

\begin{frame}[label={sec:org8e4e75e},fragile]{Paneles y grupos}
 \lstset{language=r,label= ,caption= ,captionpos=b,numbers=none}
\begin{lstlisting}
xyplot(Radiation ~ TempAvg|factor(year),
       groups=quarter,
       data=aranjuez,
       layout=c(4, 2),
       type=c('p', 'r'),
       auto.key=list(space='right'))
\end{lstlisting}
\end{frame}

\begin{frame}[label={sec:org6b297f0}]{}
\begin{center}
\includegraphics[height=\textheight]{figs/xyplotQuarterYearSmooth.png}
\end{center}
\end{frame}

\begin{frame}[label={sec:org8f8a039},fragile]{Colores y tamaños}
 \lstset{language=r,label= ,caption= ,captionpos=b,numbers=none}
\begin{lstlisting}
xyplot(Radiation ~ TempAvg,
       type=c('p', 'r'),
       cex=2, col='blue',
       alpha=.5, pch=19,
       lwd=3, col.line='black',
       data=aranjuez)
\end{lstlisting}
\end{frame}

\begin{frame}[label={sec:orgd37e0cb}]{}
\begin{center}
\includegraphics[height=\textheight]{figs/xyplotColors.png}
\end{center}
\end{frame}

\begin{frame}[label={sec:orgf696e37},fragile]{Colores con grupos}
 \lstset{language=r,label= ,caption= ,captionpos=b,numbers=none}
\begin{lstlisting}
xyplot(Radiation ~ TempAvg,
       group=quarter,
       col=c('red', 'blue', 'green', 'yellow'),
       pch=19,
       auto.key=list(space='right'),
       data=aranjuez)
\end{lstlisting}
\end{frame}

\begin{frame}[label={sec:org54c514f}]{}
\begin{center}
\includegraphics[height=\textheight]{figs/xyplotColorGroups.png}
\end{center}
\end{frame}

\begin{frame}[label={sec:org6078d79},fragile]{Colores con grupos: \texttt{par.settings} y \texttt{simpleTheme}}
 \begin{itemize}
\item Primero definimos el tema con \texttt{simpleTheme}
\end{itemize}
\lstset{language=r,label= ,caption= ,captionpos=b,numbers=none}
\begin{lstlisting}
myTheme <- simpleTheme(col=c('red', 'blue',
                             'green', 'yellow'),
                       pch=19, alpha=.6)
\end{lstlisting}
\end{frame}

\begin{frame}[label={sec:org7ab7b56},fragile]{Colores con grupos: \texttt{par.settings} y \texttt{simpleTheme}}
 \begin{itemize}
\item Aplicamos el resultado en \texttt{par.settings}
\end{itemize}
\lstset{language=r,label= ,caption= ,captionpos=b,numbers=none}
\begin{lstlisting}
xyplot(Radiation ~ TempAvg,
       groups=quarter,
       par.settings=myTheme,
       auto.key=list(space='right'),
       data=aranjuez)
\end{lstlisting}
\end{frame}

\begin{frame}[label={sec:org10fe9be}]{}
\begin{center}
\includegraphics[height=\textheight]{figs/myTheme.png}
\end{center}
\end{frame}

\begin{frame}[label={sec:org267f8c8},fragile]{Colores: brewer.pal}
 \lstset{language=r,label= ,caption= ,captionpos=b,numbers=none}
\begin{lstlisting}
library(RColorBrewer)

myPal <- brewer.pal(n = 4, 'Dark2')

myTheme <- simpleTheme(col = myPal,
                       pch=19, alpha=.6)
\end{lstlisting}

\begin{block}{ColorBrewer: \url{http://colorbrewer2.org/}}
\end{block}
\end{frame}

\begin{frame}[label={sec:orgc040022},fragile]{Asignamos paleta con \texttt{par.settings}}
 \lstset{language=r,label= ,caption= ,captionpos=b,numbers=none}
\begin{lstlisting}
xyplot(Radiation ~ TempAvg,
       groups=quarter,
       par.settings=myTheme,
       auto.key=list(space='right'),
       data=aranjuez)
\end{lstlisting}
\end{frame}

\begin{frame}[label={sec:orgac89a31}]{}
\begin{center}
\includegraphics[height=\textheight]{figs/brewer.png}
\end{center}
\end{frame}

\begin{frame}[label={sec:org2fad885},fragile]{Matriz de gráficos de dispersión}
 \lstset{language=r,label= ,caption= ,captionpos=b,numbers=none}
\begin{lstlisting}
splom(aranjuez[,c("TempAvg", "HumidAvg", "WindAvg",
                  "Rain", "Radiation", "ET")],
      pscale=0, alpha=0.6, cex=0.3, pch=19)
\end{lstlisting}

\lstset{language=r,label= ,caption= ,captionpos=b,numbers=none}
\begin{lstlisting}
library(GGally)
ggpairs(aranjuez)
\end{lstlisting}
\end{frame}

\begin{frame}[label={sec:orga146bab}]{}
\begin{center}
\includegraphics[height=\textheight]{figs/splom.png}
\end{center}
\end{frame}

\begin{frame}[label={sec:orgd77128d},fragile]{Matriz de gráficos de dispersión}
 \lstset{language=r,label= ,caption= ,captionpos=b,numbers=none}
\begin{lstlisting}
splom(aranjuez[,c("TempAvg", "HumidAvg", "WindAvg",
                  "Rain", "Radiation", "ET")],
      groups=aranjuez$quarter,
      auto.key=list(space='right'),
      pscale=0, alpha=0.6, cex=0.3, pch=19)
\end{lstlisting}
\end{frame}

\begin{frame}[label={sec:org1ec73f4}]{}
\begin{center}
\includegraphics[height=\textheight]{figs/splomGroup.png}
\end{center}
\end{frame}

\begin{frame}[label={sec:org49aa1ee},fragile]{Mapa de niveles}
 \lstset{language=r,label= ,caption= ,captionpos=b,numbers=none}
\begin{lstlisting}
levelplot(TempAvg ~ year * day, data = aranjuez)
\end{lstlisting}

\lstset{language=r,label= ,caption= ,captionpos=b,numbers=none}
\begin{lstlisting}
ggplot(aranjuez, aes(year, day)) + 
    geom_raster(aes(fill = TempAvg))
\end{lstlisting}
\end{frame}

\begin{frame}[label={sec:org9bbe135}]{}
\begin{center}
\includegraphics[height=\textheight]{figs/levelplot.png}
\end{center}
\end{frame}

\begin{frame}[label={sec:org8c489a5},fragile]{\texttt{levelplot} con una paleta mejor}
 \begin{itemize}
\item Usamos \texttt{colorRampPalette} para generar una función que interpola colores a partir de una paleta
\end{itemize}
\lstset{language=r,label= ,caption= ,captionpos=b,numbers=none}
\begin{lstlisting}
levelPal <- colorRampPalette(
    brewer.pal(n = 9, 'Oranges'))
\end{lstlisting}
\begin{itemize}
\item Comprobamos que es una función generadora de colores
\end{itemize}

\lstset{language=r,label= ,caption= ,captionpos=b,numbers=none}
\begin{lstlisting}
levelPal(14)
\end{lstlisting}

\begin{itemize}
\item Usamos esta función con \texttt{col.regions}
\end{itemize}
\lstset{language=r,label= ,caption= ,captionpos=b,numbers=none}
\begin{lstlisting}
  levelplot(TempAvg ~ year * day,
            col.regions = levelPal,
            data = aranjuez)
\end{lstlisting}
\end{frame}

\begin{frame}[label={sec:org5463c92}]{}
\begin{center}
\includegraphics[height=\textheight]{figs/levelplotPal.png}
\end{center}
\end{frame}

\begin{frame}[label={sec:org0be014f},fragile]{Gráfico de contornos}
 \lstset{language=r,label= ,caption= ,captionpos=b,numbers=none}
\begin{lstlisting}
contourplot(TempAvg ~ year * day,
            data = aranjuez,
            lwd = .5,
            labels = list(cex = 0.6),
            label.style = 'align',
            cuts = 5)
\end{lstlisting}
\end{frame}

\begin{frame}[label={sec:orgb7b5ca2}]{}
\begin{center}
\includegraphics[height=\textheight]{figs/contourplot.png}
\end{center}
\end{frame}

\begin{frame}[label={sec:org8da657a},fragile]{Box-and-Whiskers}
 \lstset{language=r,label= ,caption= ,captionpos=b,numbers=none}
\begin{lstlisting}
bwplot(Radiation ~ month, data=aranjuez,
       horizontal = FALSE, pch='|')
\end{lstlisting}

\lstset{language=r,label= ,caption= ,captionpos=b,numbers=none}
\begin{lstlisting}
ggplot(aranjuez, aes(factor(month), Radiation)) + 
    geom_boxplot()
\end{lstlisting}
\end{frame}

\begin{frame}[label={sec:orgeb97d86}]{}
\begin{center}
\includegraphics[height=\textheight]{figs/bwplot.png}
\end{center}
\end{frame}

\begin{frame}[label={sec:org2c0fcd7},fragile]{Box-and-Whiskers}
 \lstset{language=r,label= ,caption= ,captionpos=b,numbers=none}
\begin{lstlisting}
bwplot(Radiation ~ month, data=aranjuez,
       horizontal=FALSE,
       panel=panel.violin)
\end{lstlisting}

\lstset{language=r,label= ,caption= ,captionpos=b,numbers=none}
\begin{lstlisting}
ggplot(aranjuez, aes(factor(month), Radiation)) + 
    geom_violin()
\end{lstlisting}
\end{frame}

\begin{frame}[label={sec:org0fbae67}]{}
\begin{center}
\includegraphics[height=\textheight]{figs/violin.png}
\end{center}
\end{frame}


\begin{frame}[label={sec:org951f8d1},fragile]{Histogramas}
 \lstset{language=r,label= ,caption= ,captionpos=b,numbers=none}
\begin{lstlisting}
histogram(~ Radiation|factor(year), data=aranjuez)
\end{lstlisting}

\lstset{language=r,label= ,caption= ,captionpos=b,numbers=none}
\begin{lstlisting}
ggplot(aranjuez, aes(Radiation)) + 
    geom_histogram() +
    facet_wrap(~factor(year))
\end{lstlisting}
\end{frame}

\begin{frame}[label={sec:orgb6da44b}]{}
\begin{center}
\includegraphics[height=\textheight]{figs/histogram.png}
\end{center}
\end{frame}

\begin{frame}[label={sec:org966ad70},fragile]{Gráficos de densidad}
 \lstset{language=r,label= ,caption= ,captionpos=b,numbers=none}
\begin{lstlisting}
densityplot(~ Radiation, groups=quarter,
            data=aranjuez,
            auto.key=list(space='right'))
\end{lstlisting}

\lstset{language=r,label= ,caption= ,captionpos=b,numbers=none}
\begin{lstlisting}
ggplot(aranjuez, aes(Radiation, color = quarter)) + 
    geom_density()
\end{lstlisting}
\end{frame}

\begin{frame}[label={sec:orge187a1d}]{}
\begin{center}
\includegraphics[height=\textheight]{figs/density.png}
\end{center}
\end{frame}

\begin{frame}[label={sec:org7fb68ac},fragile]{Quantile-Quantile}
 \lstset{language=r,label= ,caption= ,captionpos=b,numbers=none}
\begin{lstlisting}
  firstHalf <- aranjuez$quarter %in% c('Q1', 'Q2')
  
  qq(firstHalf ~ Radiation, data=aranjuez)
\end{lstlisting}
\end{frame}

\begin{frame}[label={sec:orgba0a355}]{}
\begin{center}
\includegraphics[height=\textheight]{figs/qqHalf.png}
\end{center}
\end{frame}

\begin{frame}[label={sec:org4f770a6},fragile]{Quantile-quantile}
 \lstset{language=r,label= ,caption= ,captionpos=b,numbers=none}
\begin{lstlisting}
  winter <- aranjuez$quarter %in% c('Q1', 'Q4')
  
  qq(winter ~ Radiation, data=aranjuez)
\end{lstlisting}
\end{frame}

\begin{frame}[label={sec:org5a7cb01}]{}
\begin{center}
\includegraphics[height=\textheight]{figs/qqWinter.png}
\end{center}
\end{frame}

\begin{frame}[label={sec:org2604b37},fragile]{Quantile-Quantile}
 \lstset{language=r,label= ,caption= ,captionpos=b,numbers=none}
\begin{lstlisting}
  qqmath(~TempAvg, data=aranjuez,
         groups=year, distribution=qnorm)
\end{lstlisting}
\end{frame}

\begin{frame}[label={sec:org4170d35}]{}
\begin{center}
\includegraphics[height=\textheight]{figs/qqNorm.png}
\end{center}
\end{frame}
\end{document}