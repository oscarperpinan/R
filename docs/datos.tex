% Created 2022-06-12 dom 23:58
% Intended LaTeX compiler: pdflatex
\documentclass[aspectratio=169, usenames,svgnames,dvipsnames]{beamer}
\usepackage[utf8]{inputenc}
\usepackage[T1]{fontenc}
\usepackage{graphicx}
\usepackage{grffile}
\usepackage{longtable}
\usepackage{wrapfig}
\usepackage{rotating}
\usepackage[normalem]{ulem}
\usepackage{amsmath}
\usepackage{textcomp}
\usepackage{amssymb}
\usepackage{capt-of}
\usepackage{hyperref}
\usepackage{color}
\usepackage{listings}
\usepackage[spanish]{babel}
\setbeamercolor{alerted text}{fg=Blue}
\setbeamerfont{alerted text}{series=\bfseries}
\setbeamercolor{block title}{bg=structure.fg!20!bg!50!bg}
\setbeamercolor{block body}{use=block title,bg=block title.bg}
\AtBeginSubsection[]{\begin{frame}[plain]\tableofcontents[currentsubsection,sectionstyle=show/shaded,subsectionstyle=show/shaded/hide]\end{frame}}
\AtBeginSection[]{\begin{frame}[plain]\tableofcontents[currentsection,hideallsubsections]\end{frame}}
\lstset{keywordstyle=\color{blue}, commentstyle=\color{gray!90}, basicstyle=\ttfamily\small, columns=fullflexible, breaklines=true,linewidth=\textwidth, backgroundcolor=\color{gray!23}, basewidth={0.5em,0.4em}, literate={á}{{\'a}}1 {ñ}{{\~n}}1 {é}{{\'e}}1 {ó}{{\'o}}1 {º}{{\textordmasculine}}1, showstringspaces=false}
\usepackage{mathpazo}
\hypersetup{colorlinks=true, linkcolor=Blue, urlcolor=Blue}
\usepackage{fancyvrb}
\DefineVerbatimEnvironment{verbatim}{Verbatim}{fontsize=\tiny, formatcom = {\color{black!70}}}
\AtBeginSection[]{\begin{frame}[plain]\tableofcontents[currentsection,sectionstyle=show/shaded]\end{frame}}
\usetheme{Boadilla}
\usecolortheme{rose}
\usefonttheme{serif}
\author{\href{https://oscarperpinan.github.io}{Oscar Perpiñán Lamigueiro}}
\date{}
\title{Manejo de datos con R}
\institute[UPM]{Universidad Politécnica de Madrid}
\beamertemplatenavigationsymbolsempty
\setbeamertemplate{footline}[frame number]
\setbeamertemplate{itemize items}[triangle]
\setbeamertemplate{enumerate items}[circle]
\setbeamertemplate{section in toc}[circle]
\setbeamertemplate{subsection in toc}[circle]
\hypersetup{
 pdfauthor={\href{https://oscarperpinan.github.io}{Oscar Perpiñán Lamigueiro}},
 pdftitle={Manejo de datos con R},
 pdfkeywords={},
 pdfsubject={},
 pdfcreator={Emacs 27.1 (Org mode 9.4.6)}, 
 pdflang={Spanish}}
\begin{document}

\maketitle

\section{Lectura de datos}
\label{sec:org4f3c9b7}
\begin{frame}[label={sec:org67cb4d3},fragile]{\texttt{setwd}, \texttt{getwd}, \texttt{dir}}
 En \texttt{setwd} hay que especificar el directorio que contiene el repositorio.
\lstset{language=r,label= ,caption= ,captionpos=b,numbers=none}
\begin{lstlisting}
getwd()
old <- setwd("~/github/intro")
dir()
\end{lstlisting}

\lstset{language=r,label= ,caption= ,captionpos=b,numbers=none}
\begin{lstlisting}
dir(pattern='.R')
\end{lstlisting}

\begin{verbatim}
 [1] "birds.R"               "ClasesMetodos.R"       "datos.R"              
 [4] "estadistica.R"         "factorDateCharacter.R" "Funciones.R"          
 [7] "graficos.R"            "intro.R"               "raster.R"             
[10] "zoo.R"
\end{verbatim}


\lstset{language=r,label= ,caption= ,captionpos=b,numbers=none}
\begin{lstlisting}
dir('data')
\end{lstlisting}

\begin{verbatim}
 [1] "aranjuez.csv"        "aranjuez.RData"      "bird_tracking.csv"  
 [4] "CO2_GNI_BM.csv"      "El.Arenosillo.txt"   "eric.csv"           
 [7] "InformeDatos.zip"    "nico.csv"            "NREL-Hawaii.csv"    
[10] "radiacion_datas.csv" "sanne.csv"           "SIAR.csv"           
[13] "SISmm2008_CMSAF.zip"
\end{verbatim}
\end{frame}

\begin{frame}[label={sec:org8cb32cd},fragile]{Lectura de datos con \texttt{read.table} o \texttt{read.csv}}
 \begin{itemize}
\item Función Genérica
\end{itemize}
\lstset{language=r,label= ,caption= ,captionpos=b,numbers=none}
\begin{lstlisting}
dats <- read.table('data/aranjuez.csv', sep=',', header=TRUE)

head(dats)
\end{lstlisting}

\begin{verbatim}

           X TempAvg TempMax TempMin HumidAvg HumidMax WindAvg WindMax Rain
1 2004-01-01   4.044   10.71  -1.969     88.3     95.9   0.746   3.528    0
2 2004-01-02   5.777   11.52   1.247     83.3     98.5   1.078   6.880    0
3 2004-01-03   5.850   13.32   0.377     75.0     94.4   0.979   6.576    0
4 2004-01-04   4.408   15.59  -2.576     82.0     97.0   0.633   3.704    0
5 2004-01-05   3.081   14.58  -2.974     83.2     97.0   0.389   2.244    0
6 2004-01-06   2.304   11.83  -3.379     84.5     96.5   0.436   2.136    0
  Radiation        ET
1     5.490 0.5352688
2     6.537 0.7710499
3     8.810 0.8361229
4     9.790 0.6861381
5    10.300 0.5152422
6     9.940 0.4886631
\end{verbatim}

\begin{itemize}
\item Función específica
\end{itemize}
\lstset{language=r,label= ,caption= ,captionpos=b,numbers=none}
\begin{lstlisting}
aranjuez <- read.csv('data/aranjuez.csv')

head(aranjuez)
\end{lstlisting}

\begin{verbatim}

           X TempAvg TempMax TempMin HumidAvg HumidMax WindAvg WindMax Rain
1 2004-01-01   4.044   10.71  -1.969     88.3     95.9   0.746   3.528    0
2 2004-01-02   5.777   11.52   1.247     83.3     98.5   1.078   6.880    0
3 2004-01-03   5.850   13.32   0.377     75.0     94.4   0.979   6.576    0
4 2004-01-04   4.408   15.59  -2.576     82.0     97.0   0.633   3.704    0
5 2004-01-05   3.081   14.58  -2.974     83.2     97.0   0.389   2.244    0
6 2004-01-06   2.304   11.83  -3.379     84.5     96.5   0.436   2.136    0
  Radiation        ET
1     5.490 0.5352688
2     6.537 0.7710499
3     8.810 0.8361229
4     9.790 0.6861381
5    10.300 0.5152422
6     9.940 0.4886631
\end{verbatim}

\lstset{language=r,label= ,caption= ,captionpos=b,numbers=none}
\begin{lstlisting}
class(aranjuez)
\end{lstlisting}

\begin{verbatim}
[1] "data.frame"
\end{verbatim}
\end{frame}

\begin{frame}[label={sec:orgb38e873},fragile]{Inspeccionamos el resultado}
 \lstset{language=r,label= ,caption= ,captionpos=b,numbers=none}
\begin{lstlisting}
names(aranjuez)
\end{lstlisting}

\begin{verbatim}
[1] "X"         "TempAvg"   "TempMax"   "TempMin"   "HumidAvg"  "HumidMax" 
[7] "WindAvg"   "WindMax"   "Rain"      "Radiation" "ET"
\end{verbatim}


\lstset{language=r,label= ,caption= ,captionpos=b,numbers=none}
\begin{lstlisting}
head(aranjuez)
\end{lstlisting}

\begin{verbatim}
           X TempAvg TempMax TempMin HumidAvg HumidMax WindAvg WindMax Rain
1 2004-01-01   4.044   10.71  -1.969     88.3     95.9   0.746   3.528    0
2 2004-01-02   5.777   11.52   1.247     83.3     98.5   1.078   6.880    0
3 2004-01-03   5.850   13.32   0.377     75.0     94.4   0.979   6.576    0
4 2004-01-04   4.408   15.59  -2.576     82.0     97.0   0.633   3.704    0
5 2004-01-05   3.081   14.58  -2.974     83.2     97.0   0.389   2.244    0
6 2004-01-06   2.304   11.83  -3.379     84.5     96.5   0.436   2.136    0
  Radiation        ET
1     5.490 0.5352688
2     6.537 0.7710499
3     8.810 0.8361229
4     9.790 0.6861381
5    10.300 0.5152422
6     9.940 0.4886631
\end{verbatim}

\lstset{language=r,label= ,caption= ,captionpos=b,numbers=none}
\begin{lstlisting}
tail(aranjuez)
\end{lstlisting}

\begin{verbatim}
              X TempAvg TempMax TempMin HumidAvg HumidMax WindAvg WindMax  Rain
2893 2011-12-26   3.366   13.88  -3.397     81.5      100   0.556   3.263 0.000
2894 2011-12-27   2.222   13.33  -4.005     87.0      100   0.369   1.842 0.000
2895 2011-12-28   1.810   12.33  -4.682     85.0      100   0.540   3.401 0.203
2896 2011-12-29   2.512   11.92  -4.682     77.2      100   0.546   4.420 0.203
2897 2011-12-30   1.006   11.05  -5.822     79.7      100   0.446   2.832 0.000
2898 2011-12-31   2.263   12.67  -3.938     80.3      100   0.270   1.950 0.000
     Radiation        ET
2893      9.44 0.5358751
2894      9.52 0.4386931
2895      9.59 0.5183545
2896      9.72 0.5428373
2897      9.74 0.4614953
2898      8.11 0.4246270
\end{verbatim}
\end{frame}

\begin{frame}[label={sec:org2ee67bd},fragile]{Inspeccionamos el resultado}
 \lstset{language=r,label= ,caption= ,captionpos=b,numbers=none}
\begin{lstlisting}
summary(aranjuez)
\end{lstlisting}

\begin{verbatim}
      X                TempAvg          TempMax          TempMin       
 Length:2898        Min.   :-5.309   Min.   :-2.362   Min.   :-12.980  
 Class :character   1st Qu.: 7.692   1st Qu.:14.530   1st Qu.:  1.515  
 Mode  :character   Median :13.810   Median :21.670   Median :  7.170  
                    Mean   :14.405   Mean   :22.531   Mean   :  6.888  
                    3rd Qu.:21.615   3rd Qu.:30.875   3rd Qu.: 12.590  
                    Max.   :30.680   Max.   :41.910   Max.   : 22.710  
                                                      NA's   :4        
    HumidAvg         HumidMax         WindAvg         WindMax      
 Min.   : 19.89   Min.   : 35.88   Min.   :0.251   Min.   : 0.000  
 1st Qu.: 47.04   1st Qu.: 81.60   1st Qu.:0.667   1st Qu.: 3.783  
 Median : 62.58   Median : 90.90   Median :0.920   Median : 5.027  
 Mean   : 62.16   Mean   : 87.22   Mean   :1.174   Mean   : 5.208  
 3rd Qu.: 77.38   3rd Qu.: 94.90   3rd Qu.:1.431   3rd Qu.: 6.537  
 Max.   :100.00   Max.   :100.00   Max.   :8.260   Max.   :10.000  
                  NA's   :13       NA's   :8       NA's   :128     
      Rain          Radiation            ET       
 Min.   : 0.000   Min.   : 0.277   Min.   :0.000  
 1st Qu.: 0.000   1st Qu.: 9.370   1st Qu.:1.168  
 Median : 0.000   Median :16.660   Median :2.758  
 Mean   : 1.094   Mean   :16.742   Mean   :3.091  
 3rd Qu.: 0.200   3rd Qu.:24.650   3rd Qu.:4.926  
 Max.   :49.730   Max.   :32.740   Max.   :8.564  
 NA's   :4        NA's   :13       NA's   :18
\end{verbatim}
\end{frame}

\begin{frame}[label={sec:orgdebe588},fragile]{Valores ausentes}
 \begin{itemize}
\item \texttt{NA} está definido como \texttt{logical}
\end{itemize}
\lstset{language=r,label= ,caption= ,captionpos=b,numbers=none}
\begin{lstlisting}
class(NA)
\end{lstlisting}

\begin{verbatim}
[1] "logical"
\end{verbatim}


\begin{itemize}
\item Operar con \texttt{NA} siempre produce un \texttt{NA}
\end{itemize}
\lstset{language=r,label= ,caption= ,captionpos=b,numbers=none}
\begin{lstlisting}
1 + NA
\end{lstlisting}

\begin{verbatim}
[1] NA
\end{verbatim}


\begin{itemize}
\item Esto es un \guillemotleft{}problema\guillemotright{} al usar funciones
\end{itemize}
\lstset{language=r,label= ,caption= ,captionpos=b,numbers=none}
\begin{lstlisting}
mean(aranjuez$Radiation)
\end{lstlisting}

\begin{verbatim}
[1] NA
\end{verbatim}


\lstset{language=r,label= ,caption= ,captionpos=b,numbers=none}
\begin{lstlisting}
mean(aranjuez$Radiation, na.rm =  TRUE)
\end{lstlisting}

\begin{verbatim}
[1] 16.74176
\end{verbatim}
\end{frame}

\begin{frame}[label={sec:org4d8f3f9},fragile]{Valores ausentes}
 Las funciones \texttt{is.na} y \texttt{anyNA} los identifican 
\lstset{language=r,label= ,caption= ,captionpos=b,numbers=none}
\begin{lstlisting}
anyNA(aranjuez)
\end{lstlisting}

\begin{verbatim}
[1] TRUE
\end{verbatim}


\lstset{language=r,label= ,caption= ,captionpos=b,numbers=none}
\begin{lstlisting}
which(is.na(aranjuez$Radiation))
\end{lstlisting}

\begin{verbatim}
[1] 1861 1867 1873 1896 1897 1908 1923 2153 2413 2587 2600 2603 2684
\end{verbatim}


\lstset{language=r,label= ,caption= ,captionpos=b,numbers=none}
\begin{lstlisting}
sum(is.na(aranjuez$Radiation))
\end{lstlisting}

\begin{verbatim}
[1] 13
\end{verbatim}
\end{frame}

\begin{frame}[label={sec:org519e8fe},fragile]{Fechas}
 \lstset{language=r,label= ,caption= ,captionpos=b,numbers=none}
\begin{lstlisting}
names(aranjuez)[1] <- "Date"

aranjuez$Date <- as.Date(aranjuez$Date)
\end{lstlisting}

\lstset{language=r,label= ,caption= ,captionpos=b,numbers=none}
\begin{lstlisting}
class(aranjuez$Date)

summary(aranjuez$Date)
\end{lstlisting}

\begin{verbatim}
[1] "Date"

        Min.      1st Qu.       Median         Mean      3rd Qu.         Max. 
"2004-01-01" "2005-12-29" "2008-01-09" "2008-01-03" "2010-01-02" "2011-12-31"
\end{verbatim}
\end{frame}

\begin{frame}[label={sec:orgc513235},fragile]{Fechas}
 \begin{itemize}
\item Podemos extraer información de un objeto \texttt{Date} con la función \texttt{format}\footnote{Más información en \texttt{help(format.Date)} y \texttt{help(strptime)}.}:
\end{itemize}
\lstset{language=r,label= ,caption= ,captionpos=b,numbers=none}
\begin{lstlisting}
aranjuez$month <- as.numeric(
    format(aranjuez$Date, '%m'))

aranjuez$year <- as.numeric(
    format(aranjuez$Date, '%Y'))

aranjuez$day <- as.numeric(
    format(aranjuez$Date, '%j'))
\end{lstlisting}

\lstset{language=r,label= ,caption= ,captionpos=b,numbers=none}
\begin{lstlisting}
summary(aranjuez[, c("Date", "month", "year", "day")])
\end{lstlisting}

\begin{verbatim}
     Date                month             year           day       
Min.   :2004-01-01   Min.   : 1.000   Min.   :2004   Min.   :  1.0  
1st Qu.:2005-12-29   1st Qu.: 4.000   1st Qu.:2005   1st Qu.: 92.0  
Median :2008-01-09   Median : 7.000   Median :2008   Median :184.0  
Mean   :2008-01-03   Mean   : 6.526   Mean   :2008   Mean   :183.2  
3rd Qu.:2010-01-02   3rd Qu.:10.000   3rd Qu.:2010   3rd Qu.:274.8  
Max.   :2011-12-31   Max.   :12.000   Max.   :2011   Max.   :366.0
\end{verbatim}
\end{frame}

\section{Indexado}
\label{sec:orgb57a631}

\begin{frame}[label={sec:orgc847b36},fragile]{Indexado con \texttt{[]}}
 \begin{itemize}
\item Filas
\end{itemize}
\lstset{language=r,label= ,caption= ,captionpos=b,numbers=none}
\begin{lstlisting}
aranjuez[1:5,]
\end{lstlisting}

\begin{verbatim}
        Date TempAvg TempMax TempMin HumidAvg HumidMax WindAvg WindMax Rain
1 2004-01-01   4.044   10.71  -1.969     88.3     95.9   0.746   3.528    0
2 2004-01-02   5.777   11.52   1.247     83.3     98.5   1.078   6.880    0
3 2004-01-03   5.850   13.32   0.377     75.0     94.4   0.979   6.576    0
4 2004-01-04   4.408   15.59  -2.576     82.0     97.0   0.633   3.704    0
5 2004-01-05   3.081   14.58  -2.974     83.2     97.0   0.389   2.244    0
  Radiation        ET month year day
1     5.490 0.5352688     1 2004   1
2     6.537 0.7710499     1 2004   2
3     8.810 0.8361229     1 2004   3
4     9.790 0.6861381     1 2004   4
5    10.300 0.5152422     1 2004   5
\end{verbatim}

\begin{itemize}
\item Filas y Columnas
\end{itemize}
\lstset{language=r,label= ,caption= ,captionpos=b,numbers=none}
\begin{lstlisting}
aranjuez[10:14, 1:5]
\end{lstlisting}

\begin{verbatim}
         Date TempAvg TempMax TempMin HumidAvg
10 2004-01-10   10.85   16.59   5.676     84.9
11 2004-01-11    7.59    9.23   4.806     95.4
12 2004-01-12    7.41   10.24   5.200     93.1
13 2004-01-13    8.35   11.38   4.137     91.3
14 2004-01-14    8.74   13.32   2.857     86.9
\end{verbatim}
\end{frame}

\begin{frame}[label={sec:org300ffbb},fragile]{Indexado con \texttt{[]}}
 \begin{itemize}
\item Condición basada en los datos
\end{itemize}
\lstset{language=r,label= ,caption= ,captionpos=b,numbers=none}
\begin{lstlisting}
idx <- with(aranjuez, Radiation > 20 & TempAvg < 10) 

head(aranjuez[idx, ])
\end{lstlisting}

\begin{verbatim}

          Date TempAvg TempMax TempMin HumidAvg HumidMax WindAvg WindMax Rain
82  2004-03-22    9.78   16.12   4.340    51.65     87.9   1.526   7.660    0
83  2004-03-23    8.50   15.52  -0.290    50.10     83.3   1.533   6.027    0
85  2004-03-25    7.47   14.58   1.584    49.66     76.6   1.138   5.939    0
100 2004-04-09    8.83   15.52   2.056    47.50     70.8   1.547   6.125    0
101 2004-04-10    7.04   13.85  -0.155    54.45     85.8   1.448   6.958    0
102 2004-04-11    7.50   15.19  -1.699    54.98     91.0   1.126   7.590    0
    Radiation       ET month year day
82      21.92 3.075785     3 2004  82
83      20.62 2.881419     3 2004  83
85      22.44 2.849603     3 2004  85
100     25.45 3.566452     4 2004 100
101     21.07 2.943239     4 2004 101
102     20.99 2.905479     4 2004 102
\end{verbatim}
\end{frame}

\begin{frame}[label={sec:org38fe785},fragile]{\texttt{subset}}
 \lstset{language=r,label= ,caption= ,captionpos=b,numbers=none}
\begin{lstlisting}
subset(aranjuez,
       subset = (Radiation > 20 & TempAvg < 10),
       select = c(Radiation, TempAvg,
           TempMax, TempMin))
\end{lstlisting}

\begin{verbatim}

     Radiation TempAvg TempMax TempMin
82       21.92   9.780   16.12   4.340
83       20.62   8.500   15.52  -0.290
85       22.44   7.470   14.58   1.584
100      25.45   8.830   15.52   2.056
101      21.07   7.040   13.85  -0.155
102      20.99   7.500   15.19  -1.699
104      25.76   9.420   17.47   0.115
461      24.29   7.460   14.66  -0.081
462      25.25   7.930   17.35  -1.686
463      24.56   9.800   19.08  -1.484
1146     20.08   7.170   18.20  -3.746
1157     20.90   4.378   12.03  -6.353
1159     21.87   7.920   18.54  -2.941
1160     20.35   7.830   16.49  -2.807
1521     21.54   8.100   19.29  -4.075
2244     20.49   6.121   15.15  -0.940
2245     21.02   5.989   16.94  -3.208
2246     20.22   9.020   19.74  -2.068
2261     23.00   9.500   14.96   3.662
2262     20.40   9.910   14.70   4.668
2263     24.09   9.440   16.89   0.794
2265     23.64   9.680   16.35   2.938
2295     22.46   8.730   13.84   1.740
\end{verbatim}
\end{frame}

\begin{frame}[label={sec:orga8e3994},fragile]{Ejercicio}
 \begin{block}{Valores en las estaciones}
Extrae dos subconjuntos de datos, uno correspondiente al invierno y otro correspondiente al verano, incluyendo las variables de radiación y temperatura media, fecha y mes. 

Con estos dos \texttt{data.frame} obtén uno conjunto, diferenciando la estación de cada registro.

Puedes suponer que el invierno comenzó el 22 de diciembre y terminó el 20 de marzo, y el verano comenzó el 21 de junio y terminó el 23 de septiembre.
\end{block}
\end{frame}


\begin{frame}[label={sec:orgd966bef},fragile]{Solución}
 \lstset{language=r,label= ,caption= ,captionpos=b,numbers=none}
\begin{lstlisting}
invierno <- subset(aranjuez,
                   select = c(Date, day, month, 
                              Radiation, TempAvg),
                   subset = day < 79 | day > 357)
\end{lstlisting}

\lstset{language=r,label= ,caption= ,captionpos=b,numbers=none}
\begin{lstlisting}
verano <- subset(aranjuez,
                 select = c(Date, day, month,
                            Radiation, TempAvg),
                   subset = day > 173 & day < 267)
\end{lstlisting}

\lstset{language=r,label= ,caption= ,captionpos=b,numbers=none}
\begin{lstlisting}
invierno$id <- "Invierno"
verano$id <- "Verano"

aranjuez2 <- rbind(invierno, verano)
\end{lstlisting}
\end{frame}


\section{Datos agregados}
\label{sec:org67886a3}

\begin{frame}[label={sec:org09343a5},fragile]{\texttt{aggregate}}
 \lstset{language=r,label= ,caption= ,captionpos=b,numbers=none}
\begin{lstlisting}
aranjuez$rainy <- aranjuez$Rain > 0

aggregate(Radiation ~ rainy, data = aranjuez,
          FUN = mean)
\end{lstlisting}

\begin{verbatim}

  rainy Radiation
1 FALSE  19.63325
2  TRUE  10.26028
\end{verbatim}
\end{frame}

\begin{frame}[label={sec:orgf02d5be},fragile]{Variable categórica con \texttt{cut}}
 \lstset{language=r,label= ,caption= ,captionpos=b,numbers=none}
\begin{lstlisting}
aranjuez$tempClass <- cut(aranjuez$TempAvg, 5)

aggregate(Radiation ~ tempClass, data = aranjuez,
          FUN = mean)
\end{lstlisting}

\begin{verbatim}

     tempClass Radiation
1 (-5.34,1.89]  8.805389
2  (1.89,9.09]  9.014178
3  (9.09,16.3] 14.554177
4  (16.3,23.5] 21.912414
5  (23.5,30.7] 26.192742
\end{verbatim}


\lstset{language=r,label= ,caption= ,captionpos=b,numbers=none}
\begin{lstlisting}
aggregate(Radiation ~ tempClass + rainy,
          data = aranjuez, FUN = mean)
\end{lstlisting}

\begin{verbatim}

      tempClass rainy Radiation
1  (-5.34,1.89] FALSE  9.869134
2   (1.89,9.09] FALSE 10.718837
3   (9.09,16.3] FALSE 17.238283
4   (16.3,23.5] FALSE 23.238145
5   (23.5,30.7] FALSE 26.392665
6  (-5.34,1.89]  TRUE  6.822955
7   (1.89,9.09]  TRUE  7.063932
8   (9.09,16.3]  TRUE 11.091063
9   (16.3,23.5]  TRUE 15.802522
10  (23.5,30.7]  TRUE 22.545862
\end{verbatim}
\end{frame}

\begin{frame}[label={sec:orgb388a58},fragile]{Agregamos varias variables}
 \lstset{language=r,label= ,caption= ,captionpos=b,numbers=none}
\begin{lstlisting}
aggregate(cbind(Radiation, TempAvg) ~ tempClass,
          data = aranjuez, FUN = mean)
\end{lstlisting}

\begin{verbatim}

     tempClass Radiation    TempAvg
1 (-5.34,1.89]  8.805389  0.3423095
2  (1.89,9.09]  9.014178  5.6663267
3  (9.09,16.3] 14.554177 12.5219084
4  (16.3,23.5] 21.912414 19.7486310
5  (23.5,30.7] 26.192742 26.0496953
\end{verbatim}


\lstset{language=r,label= ,caption= ,captionpos=b,numbers=none}
\begin{lstlisting}
aggregate(cbind(Radiation, TempAvg) ~ tempClass + rainy,
          data = aranjuez, FUN = mean)
\end{lstlisting}

\begin{verbatim}

      tempClass rainy Radiation    TempAvg
1  (-5.34,1.89] FALSE  9.869134  0.3550122
2   (1.89,9.09] FALSE 10.718837  5.6657481
3   (9.09,16.3] FALSE 17.238283 12.6959488
4   (16.3,23.5] FALSE 23.238145 19.9486604
5   (23.5,30.7] FALSE 26.392665 26.0896408
6  (-5.34,1.89]  TRUE  6.822955  0.3186364
7   (1.89,9.09]  TRUE  7.063932  5.6669887
8   (9.09,16.3]  TRUE 11.091063 12.2973563
9   (16.3,23.5]  TRUE 15.802522 18.8267565
10  (23.5,30.7]  TRUE 22.545862 25.3210345
\end{verbatim}
\end{frame}


\begin{frame}[label={sec:orga07a239},fragile]{Ejercicio}
 \begin{block}{Valores en las estaciones}
A partir del \texttt{data.frame} que incluía los datos de invierno y verano, calcula:

\begin{itemize}
\item La \alert{mediana} de las variables de radiación y temperatura por estación.
\item La \alert{desviación estándar} relativa a la media de las variables de radiación y temperatura por estación.
\end{itemize}

A partir del \texttt{data.frame} completo calcula la \alert{media} interanual diaria de las variables de radiación y temperatura.
\end{block}
\end{frame}

\begin{frame}[label={sec:org7b6a550},fragile]{Solución}
 \lstset{language=r,label= ,caption= ,captionpos=b,numbers=none}
\begin{lstlisting}
## Mediana
aggregate(cbind(Radiation, TempAvg) ~ id,
          data = aranjuez2,
          FUN = median)

## Desviación estándar relativa
sdr <- function(x) sd(x) / mean(x)

aggregate(cbind(Radiation, TempAvg) ~ id,
          data = aranjuez2,
          FUN = sdr)

## Media interanual
aggregate(cbind(Radiation, TempAvg) ~ day,
          data = aranjuez,
          FUN = mean)
\end{lstlisting}
\end{frame}

\section{Unión de \texttt{data.frame}}
\label{sec:orgeb8cabf}
\begin{frame}[label={sec:org8b8852d},fragile]{Con \texttt{merge}}
 \begin{itemize}
\item Primero construimos un \texttt{data.frame} de ejemplo
\end{itemize}
\lstset{language=r,label= ,caption= ,captionpos=b,numbers=none}
\begin{lstlisting}
USStates <- as.data.frame(state.x77)
USStates$Name <- rownames(USStates)
rownames(USStates) <- NULL
\end{lstlisting}

\begin{itemize}
\item Lo partimos en estados \guillemotleft{}fríos\guillemotright{} y estados \guillemotleft{}grandes\guillemotright{}
\end{itemize}
\lstset{language=r,label= ,caption= ,captionpos=b,numbers=none}
\begin{lstlisting}
coldStates <- USStates[USStates$Frost>150,
                       c('Name', 'Frost')]
largeStates <- USStates[USStates$Area>1e5,
                        c('Name', 'Area')]
\end{lstlisting}
\end{frame}

\begin{frame}[label={sec:org9555fd0},fragile]{Con \texttt{merge}}
 \begin{itemize}
\item Unimos los dos conjuntos (estados \guillemotleft{}fríos\guillemotright{} y \guillemotleft{}grandes\guillemotright{})
\end{itemize}
\lstset{language=r,label= ,caption= ,captionpos=b,numbers=none}
\begin{lstlisting}
merge(coldStates, largeStates)
\end{lstlisting}

\begin{verbatim}
      Name Frost   Area
1   Alaska   152 566432
2 Colorado   166 103766
3  Montana   155 145587
4   Nevada   188 109889
\end{verbatim}
\end{frame}

\begin{frame}[label={sec:org0daf063},fragile]{\texttt{merge} usa \texttt{match}}
 \begin{itemize}
\item Estados grandes que también son fríos
\end{itemize}
\lstset{language=r,label= ,caption= ,captionpos=b,numbers=none}
\begin{lstlisting}
idxLarge <- match(largeStates$Name,
                  coldStates$Name,
                  nomatch=0)
idxLarge
\end{lstlisting}

\begin{verbatim}

[1] 1 0 0 2 5 6 0 0
\end{verbatim}


\lstset{language=r,label= ,caption= ,captionpos=b,numbers=none}
\begin{lstlisting}
coldStates[idxLarge,]
\end{lstlisting}

\begin{verbatim}
       Name Frost
2    Alaska   152
6  Colorado   166
26  Montana   155
28   Nevada   188
\end{verbatim}
\end{frame}

\begin{frame}[label={sec:org4713eaa},fragile]{\texttt{merge} usa \texttt{match}}
 \begin{itemize}
\item Estados frios que también son grandes
\end{itemize}
\lstset{language=r,label= ,caption= ,captionpos=b,numbers=none}
\begin{lstlisting}
idxCold <- match(coldStates$Name,
                 largeStates$Name,
                 nomatch=0)
idxCold
\end{lstlisting}

\begin{verbatim}

[1] 1 4 0 0 5 6 0 0 0 0 0
\end{verbatim}


\lstset{language=r,label= ,caption= ,captionpos=b,numbers=none}
\begin{lstlisting}
largeStates[idxCold,]
\end{lstlisting}

\begin{verbatim}
       Name   Area
2    Alaska 566432
6  Colorado 103766
26  Montana 145587
28   Nevada 109889
\end{verbatim}
\end{frame}

\section{Cambio de formato}
\label{sec:org34f484c}
\begin{frame}[label={sec:org8d3688d},fragile]{Forma simple con \texttt{stack}}
 \lstset{language=r,label= ,caption= ,captionpos=b,numbers=none}
\begin{lstlisting}
aranjuezWide <- aranjuez[, c('Date', 'Radiation',
                             'TempAvg', 'TempMax',
                             'WindAvg', 'WindMax')]
\end{lstlisting}

\begin{itemize}
\item Pasamos de formato \texttt{wide} a \texttt{long}
\end{itemize}
\lstset{language=r,label= ,caption= ,captionpos=b,numbers=none}
\begin{lstlisting}
aranjuezLong <- stack(aranjuezWide)

head(aranjuezLong)
\end{lstlisting}

\begin{verbatim}
Warning message:
In stack.data.frame(aranjuezWide) : non-vector columns will be ignored

  values       ind
1  5.490 Radiation
2  6.537 Radiation
3  8.810 Radiation
4  9.790 Radiation
5 10.300 Radiation
6  9.940 Radiation
\end{verbatim}

\lstset{language=r,label= ,caption= ,captionpos=b,numbers=none}
\begin{lstlisting}
summary(aranjuezLong)
\end{lstlisting}

\begin{verbatim}
    values              ind      
Min.   :-5.309   Radiation:2898  
1st Qu.: 3.158   TempAvg  :2898  
Median : 8.720   TempMax  :2898  
Mean   :12.074   WindAvg  :2898  
3rd Qu.:19.970   WindMax  :2898  
Max.   :41.910                   
NA's   :149
\end{verbatim}
\end{frame}

\begin{frame}[label={sec:org23fa577},fragile]{Más flexible con \texttt{reshape2}}
 \begin{itemize}
\item \texttt{reshape2} es un paquete que puede facilitar la transformación de \texttt{data.frame} y matrices.
\end{itemize}

\lstset{language=r,label= ,caption= ,captionpos=b,numbers=none}
\begin{lstlisting}
library(reshape2)
\end{lstlisting}
\end{frame}

\begin{frame}[label={sec:org39caf1d},fragile]{\texttt{melt} para cambiar de \emph{wide} a \emph{long}}
 \lstset{language=r,label= ,caption= ,captionpos=b,numbers=none}
\begin{lstlisting}
aranjuezLong2 <- melt(aranjuezWide, id.vars = 'Date',
                      variable.name = 'Variable',
                      value.name = 'Value')

head(aranjuezLong2)
\end{lstlisting}

\begin{verbatim}

        Date  Variable  Value
1 2004-01-01 Radiation  5.490
2 2004-01-02 Radiation  6.537
3 2004-01-03 Radiation  8.810
4 2004-01-04 Radiation  9.790
5 2004-01-05 Radiation 10.300
6 2004-01-06 Radiation  9.940
\end{verbatim}
\end{frame}

\begin{frame}[label={sec:org0b22ea1},fragile]{Agregamos a partir de un formato \texttt{long}}
 \lstset{language=r,label= ,caption= ,captionpos=b,numbers=none}
\begin{lstlisting}
aggregate(Value ~ Variable, data = aranjuezLong2,
          FUN = mean)
\end{lstlisting}

\begin{verbatim}

   Variable     Value
1 Radiation 16.741759
2   TempAvg 14.404856
3   TempMax 22.531033
4   WindAvg  1.173983
5   WindMax  5.208021
\end{verbatim}
\end{frame}

\begin{frame}[label={sec:org44498a5},fragile]{\texttt{dcast} para cambiar de \emph{long} a \emph{wide}}
 \lstset{language=r,label= ,caption= ,captionpos=b,numbers=none}
\begin{lstlisting}
aranjuezWide2 <- dcast(aranjuezLong2,
                       Variable ~ Date)
head(aranjuezWide2[, 1:10])
\end{lstlisting}

\begin{verbatim}

Using Value as value column: use value.var to override.

   Variable 2004-01-01 2004-01-02 2004-01-03 2004-01-04 2004-01-05 2004-01-06
1 Radiation      5.490      6.537      8.810      9.790     10.300      9.940
2   TempAvg      4.044      5.777      5.850      4.408      3.081      2.304
3   TempMax     10.710     11.520     13.320     15.590     14.580     11.830
4   WindAvg      0.746      1.078      0.979      0.633      0.389      0.436
5   WindMax      3.528      6.880      6.576      3.704      2.244      2.136
  2004-01-07 2004-01-08 2004-01-09
1      7.410      4.630      4.995
2      2.080      6.405     12.060
3     11.500     13.380     15.330
4      0.449      1.188      2.737
5      3.949      6.821      7.750
\end{verbatim}
\end{frame}
\end{document}